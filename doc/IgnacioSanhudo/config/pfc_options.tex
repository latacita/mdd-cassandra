%
% Potentially useful packages (rec = recommended, opt = optional)
%
\usepackage{fancyhdr}          % (rec)  allows for the customization of various header/footer parameters
% \usepackage{courier}         % (opt)  uses that font by default
% \usepackage{setspace}        % (opt)  allows for inter-line space changing
\usepackage{longtable}         % (opt)  allows for multi-page tables
% \usepackage{lscape}          % (opt)  allows for the use of \landscape
\usepackage{color}             % (opt)  various color-related commands (like \color)
\usepackage{rotating}          % (opt)  allows for PS and EPS rotation
% \usepackage{textcomp}        % (opt)  allows for euro sign, with \texteuro
\usepackage[spanish]{minitoc}           % (opt)  allows for per-chapter tables of contents
\usepackage{epsf}              % (opt)  allow for certain EPS manipulations
%\usepackage[utf8x]{inputenc}  % (opt)  allows for some text editors to show \'{a} as �, and so on.
\usepackage[absolute]{textpos} % (rec)  allows for arbitrary positioning of text (required for default cover page)
% \usepackage{srcltx}            % (opt)  allows to pass from .dvi back to the .tex
%
% Margin settings. Uncomment and modify if you know what you are doing. Note
% that a further 1 inch is added to the margins given here. The given values are
% the default ones for A4 paper, and itsas_pfc.cls style.
%
% \setlength{\oddsidemargin}{1inch}     % left margin for odd (right) pages
% \setlength{\evensidemargin}{52pt}    % left margin for even (left) pages
%\setlength{\textwidth}{390pt}        % width of the text body

%
% Recommended to improve the automatic positioning of figures.
% (taken from http://dcwww.camp.dtu.dk/~schiotz/comp/LatexTips/LatexTips.html#captfont)
%
\renewcommand{\topfraction}{0.85}
\renewcommand{\textfraction}{0.1}
\renewcommand{\floatpagefraction}{0.75}

%
% Space between top border of page and where text begins (headers go there)
% LaTeX complains if using package fancyhdr and headheight is below 15pt
%
\headheight 15pt

%
% For the textpos package (used when making the cover page)
%
\setlength{\TPHorizModule}{\paperwidth}
\setlength{\TPVertModule}{\paperheight}
\newcommand{\tb}[4]{\begin{textblock}{#1}[0.5,0.5](#2,#3)\begin{center}#4\end{center}\end{textblock}}

%
% You can define your commands here
%
% \newcommand{cmd}[args]{def}
% cmd  = command to define (e.g. \water)
% args = number of arguments
% def  = the definition, where #1, #2,... is the 1st, 2nd... argument
%
% E.g.:
%
% \newcommand{\water}[1]{H\ensuremath{_#1}O}
%
%
%
% Each time we write "\water{33}", the output will be: "H33O" (with 33 subscripted)
%

%
% You can teach LaTeX how to hypenize some words here
% E.g.: to cut "gnomonly" only where dashed (-).
%
\hyphenation{gno-mon-ly}

%
% Start book with Roman-numbered pages
% Will be changed to arabics later on
%
\pagenumbering{Roman}
