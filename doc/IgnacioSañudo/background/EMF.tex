%%==================================================================%%
%% Author : Sa�udo Olmedo, Ignacio                                  %%
%%          S�nchez Barreiro, Pablo                                 %%
%% Version: 1.3, 18/06/2014                                         %%
%%                                                                  %%
%% Memoria del Proyecto Fin de Carrera                              %%
%% Background/EMF                                                   %%
%===================================================================%%

%%======================================================================%%
%% NOTA(Pablo): Esto es paja, hay que ir un poco m�s directo al grano   %%
%%======================================================================%%
%% 
%% Para comenzar el desarrollo del proyecto bajo el paradigma del 
%% desarrollo software dirigido por modelos necesitamos definir que 
%% lenguajes de modelado vamos a utilizar. En el cap�tulo anterior 
%% coment�bamos que es necesario definir el lenguaje de modelado que 
%% utilizara Cassandra para ello usaremos EMF. 

Eclipse Modeling Framework (EMF)~\cite{dave:2008} es un framework de modelado que nos proporciona la base para la elaboraci�n de lenguajes de modelado. Para ello proporciona un lenguaje de metamodelado denominado Ecore, que puede considerarse un subconjunto de los diagramas de clase UML. Por ello, utilizando EMF se pueden crear metamodelos de forma gr�fica muy similar a los diagramas de clases en UML.

%%================================================================================%%
%% NOTA(Pablo): Volver a poner el metamodelo del grafo del cap�tulo anterior como %% 
%%              decir que viene del cap�tulo anterior y no describirlo mucho      %%
%%================================================================================%%

%%================================================================================%%
%% NOTA(Pablo): El genmodel en realidad no se utiliza en todo el proyecto,        %%
%%              por lo que no es necesario describirlo                            %% 
%%================================================================================%%
%% Para la creaci�n de meta-modelos EMF (\cite{kolovos:2014}) utiliza dos modelos 
%% de meta-datos: \emph{Ecore} y \emph{Genmodel}. Ecore contiene la informaci�n 
%% sobre las clases que se han definido. Genmodel contiene informaci�n adicional 
%% para la generaci�n del c�digo, por ejemplo la ruta y la informaci�n del archivo. 
%% Genmodel tambi�n contiene atributos que sirven de control a la hora de generar 
%% el c�digo, encontramos los siguientes par�metros de control:
%% \begin{enumerate}
%% \item EClass: representa una clase, con cero o m�s atributos y cero o m�s 
%% referencias.
%% \item EAttribute: representa un atributo que tiene un nombre y un tipo.
%% \item EReference: representa un extremo de una asociaci�n entre dos clases.
%% \item EDataType: representa el tipo de un atributo, por ejemplo, int, float.
%% 
%% \end{enumerate}
%% Estos atributos nos servir�n a la hora de crear las reglas de transformaci�n entre 
%% modelos UML y modelos Cassandra.

%%================================================================================%%
%% NOTA(Pablo): Esto es paja                                                      %%
%%================================================================================%%
%%
%% EMF permite crear un meta-modelo a trav�s de diferentes medios, por ejemplo, 
%% XMI,anotaciones Java, XML o UML. Adem�s EMF proporciona un framework para 
%% almacenar la informaci�n del modelo.
%%
%%================================================================================%%

%%================================================================================%%
%% NOTA(Pablo): Esto no aporta nada al proyecto, as� que se elimina
%%================================================================================%%
%% 
%% Un ejemplo sencillo de la utilidad de EMF es el siguiente: Imaginemos que 
%% deseamos construir una aplicaci�n para manipular mensajes escritos en XML. El 
%% primer paso que dar�amos ser�a empezar definiendo el schema del mensaje sin 
%% embargo con EMF se puede trabajar ignorando este nivel. Con EMF podemos crear 
%% plugins que generen por ejemplo un diagrama de clases UML a partir de este 
%% mensaje XML o directamente generar el c�digo Java que implemente las clases 
%% del mensaje XML.
%% 
%%================================================================================%%

Una vez que hemos definido un metamodelo en Ecore, es posible crear instancias del mismo, a�n no disponiendo de sintaxis concreta para el mismo. Adem�s, EMF permite generar una serie de clases Java para la manipulaci�n de dicho modelo, un esquema XML para poder persistir los modelos creados y las facilidades necesarias para poder crear una instancia de dicho esquema para un modelo dado. 

Con el paso de los a�os, EMF y Ecore se han convertido en los est�ndares de facto para la ingenier�a software dirigida por modelos. De esta forma, cada nueva herramienta de modelado que ha ido surgiendo, no s�lo es compatible con Ecore, sino que exige que los modelos que manipula posean un metamodelo definido en Ecore. Por ejemplo, pr�cticamente la totalidad de los lenguajes de transformaci�n de modelos s�lo aceptan como entrada y generan como salida modelos conformes a metamodelos definidos en Ecore. En adelante, nos referiremos a los modelos conformes a metamodelos definidos en Ecore simplemente como modelos Ecore, por simplicidad.  

Dentro de este Proyecto Fin de Carrera, Ecore se ha utilizado para definir un metamodelo para Cassandra, que sirva como representaci�n intermedia del c�digo Cassandra y que facilite su posterior generaci�n.

