%%==================================================================%%
%% Author : Sa�udo Olmedo, Ignacio                                  %%
%%          S�nchez Barreiro, Pablo                                 %%
%% Version: 1.3, 18/06/2013                                         %%
%%                                                                  %%
%% Memoria del Proyecto Fin de Carrera                              %%
%% Antecedentes, archivo ra�z                                       %%
%%==================================================================%%

\chapterheader{Antecedentes y Planificaci�n}{Antecedentes y Planificaci�n}
\label{chap:background}

Este cap�tulo describe las tecnolog�as y t�cnicas utilizadas en el desarrollo del presente Proyecto de Fin de Carrera. La primera secci�n est� dedicada a introducir el caso de estudio del PFC presentado, el cual consiste en la realizaci�n de un generador de c�digo Cassandra que transforma modelos UML a modelos escritos en Cassandra. La siguiente secci�n est� dedicada a describir en qu� consiste EMF el lenguaje utilizado para desarrollar lenguajes de modelado.  A continuaci�n se describe la herramienta Epsilon que se ha utilizado para desarrollar el generador de c�digo. La siguiente secci�n est� dedicada a explicar de manera breve algunos conceptos de Cassandra. Finalmente se expone la planificaci�n que ha seguido el proyecto para su realizaci�n, desde formaci�n hasta desarrollo.

\chaptertoc

\section{Caso de estudio: Generador de c�digo CQL-Cassandra}
\label{sec:back:casoEstudio}

%%==================================================================%%
%% Author : Sa�udo Olmedo, Ignacio                                  %%
%%          S�nchez Barreiro, Pablo                                 %%
%% Version: 1.2, 18/06/2014                                         %%
%%                                                                  %%
%% Memoria del Proyecto Fin de Carrera                              %%
%% Planificacion/CasoEstudio                                        %%
%%==================================================================%%

\begin{figure}[!tb]
  \centering
  \includegraphics[width=.8\linewidth]{m2m/images/twissandra.eps} \\
  \caption{Modelo UML Twissandra}
  \label{back:fig:twissandra}
\end{figure}

Esta secci�n presenta \emph{Twissandra}, el caso de estudio que se utilizar� lo largo de este proyecto. Twissandra es un proyecto creado para aprender como utilizar Cassandra. El modelo UML correspondiente a Twissandra se puede ver en la figura~\ref{back:fig:twissandra}. Twissandra es una versi�n simplificada de Twitter.

Twitter\footnote{} es una red social de microblogging, actualmente est� muy extendida, que permite escribir a sus usuarios peque�os mensajes de texto, denominados \emph{tweets}. Un \emph{tweet} es simplemente un texto con un l�mite de 140 caracteres publicado a una hora determinada. La colecci�n de todos los tweets publicados por un usuario cronol�gicamente ordenados determinan su \emph{userline}.

Cada usuario registrado en Twitter puede seguir a otros usuarios registrados. Cuando decimos, por ejemplo, que Pedro sigue a Mar�a significa que Pedro recibir� en su cuenta todos los mensajes que publique Mar�a. En este caso, se dice que Pedro es un \emph{follower} de Mar�a. De esta forma, cada usuario tiene asociado un \emph{timeline} que no es m�s que la colecci�n de todos los \emph{tweets} publicados por las personas a las que sigue ordenados cronol�gicamente, es decir, por fecha de publicaci�n. 

Los principales casos de uso de \emph{Twissandra} son: 

\begin{enumerate}
    \item Obtener el \emph{timeline} de un usuario determinado. 
    \item Obtener el \emph{userline} de un usuario determinado. 
    \item Obtener la lista de usuarios que un usuario est� siguiendo.  
    \item Obtener la lista de usuarios que est�n siguiendo a un usuario espec�fico. 
\end{enumerate}

Las dos �ltimas listas deben se deben devolver ordenadas cronol�gicamente.

%%========================================================================================%%
%% NOTA(Pablo): Esta justificaci�n es d�bil, as� que mejor quitarla                       %%
%%========================================================================================%%

Como vemos \imp{Twissandra} es una versi�n simplificada de Twitter. Sin embargo, se considera un buen ejemplo a analizar ya que Twitter goza de gran popularidad y es una de las redes sociales m�s usadas del mundo. 

Una vez entendido esto en los siguientes cap�tulos se procede a detallar el proceso de creaci�n de un repositorio de datos en Cassandra que cubra los casos de usos citados anteriormente describiendo los procesos de transformaci�n entre modelos que se realizaran as� como la generaci�n de c�digo del repositorio de datos en Twissandra.


\section{EMF}
\label{sec:back:spl}

%%==================================================================%%
%% Author : Sa�udo Olmedo, Ignacio                                  %%
%%          S�nchez Barreiro, Pablo                                 %%
%% Version: 1.3, 18/06/2014                                         %%
%%                                                                  %%
%% Memoria del Proyecto Fin de Carrera                              %%
%% Background/EMF                                                   %%
%===================================================================%%

%%======================================================================%%
%% NOTA(Pablo): Esto es paja, hay que ir un poco m�s directo al grano   %%
%%======================================================================%%
%% 
%% Para comenzar el desarrollo del proyecto bajo el paradigma del 
%% desarrollo software dirigido por modelos necesitamos definir que 
%% lenguajes de modelado vamos a utilizar. En el cap�tulo anterior 
%% coment�bamos que es necesario definir el lenguaje de modelado que 
%% utilizara Cassandra para ello usaremos EMF. 

Eclipse Modeling Framework (EMF)~\cite{dave:2008} es un framework de modelado que nos proporciona la base para la elaboraci�n de lenguajes de modelado. Para ello proporciona un lenguaje de metamodelado denominado Ecore, que puede considerarse un subconjunto de los diagramas de clase UML. Por ello, utilizando EMF se pueden crear metamodelos de forma gr�fica muy similar a los diagramas de clases en UML.

%%================================================================================%%
%% NOTA(Pablo): Volver a poner el metamodelo del grafo del cap�tulo anterior como %% 
%%              decir que viene del cap�tulo anterior y no describirlo mucho      %%
%%================================================================================%%

%%================================================================================%%
%% NOTA(Pablo): El genmodel en realidad no se utiliza en todo el proyecto,        %%
%%              por lo que no es necesario describirlo                            %% 
%%================================================================================%%
%% Para la creaci�n de meta-modelos EMF (\cite{kolovos:2014}) utiliza dos modelos 
%% de meta-datos: \emph{Ecore} y \emph{Genmodel}. Ecore contiene la informaci�n 
%% sobre las clases que se han definido. Genmodel contiene informaci�n adicional 
%% para la generaci�n del c�digo, por ejemplo la ruta y la informaci�n del archivo. 
%% Genmodel tambi�n contiene atributos que sirven de control a la hora de generar 
%% el c�digo, encontramos los siguientes par�metros de control:
%% \begin{enumerate}
%% \item EClass: representa una clase, con cero o m�s atributos y cero o m�s 
%% referencias.
%% \item EAttribute: representa un atributo que tiene un nombre y un tipo.
%% \item EReference: representa un extremo de una asociaci�n entre dos clases.
%% \item EDataType: representa el tipo de un atributo, por ejemplo, int, float.
%% 
%% \end{enumerate}
%% Estos atributos nos servir�n a la hora de crear las reglas de transformaci�n entre 
%% modelos UML y modelos Cassandra.

%%================================================================================%%
%% NOTA(Pablo): Esto es paja                                                      %%
%%================================================================================%%
%%
%% EMF permite crear un meta-modelo a trav�s de diferentes medios, por ejemplo, 
%% XMI,anotaciones Java, XML o UML. Adem�s EMF proporciona un framework para 
%% almacenar la informaci�n del modelo.
%%
%%================================================================================%%

%%================================================================================%%
%% NOTA(Pablo): Esto no aporta nada al proyecto, as� que se elimina
%%================================================================================%%
%% 
%% Un ejemplo sencillo de la utilidad de EMF es el siguiente: Imaginemos que 
%% deseamos construir una aplicaci�n para manipular mensajes escritos en XML. El 
%% primer paso que dar�amos ser�a empezar definiendo el schema del mensaje sin 
%% embargo con EMF se puede trabajar ignorando este nivel. Con EMF podemos crear 
%% plugins que generen por ejemplo un diagrama de clases UML a partir de este 
%% mensaje XML o directamente generar el c�digo Java que implemente las clases 
%% del mensaje XML.
%% 
%%================================================================================%%

Una vez que hemos definido un metamodelo en Ecore, es posible crear instancias del mismo, a�n no disponiendo de sintaxis concreta para el mismo. Adem�s, EMF permite generar una serie de clases Java para la manipulaci�n de dicho modelo, un esquema XML para poder persistir los modelos creados y las facilidades necesarias para poder crear una instancia de dicho esquema para un modelo dado. 

Con el paso de los a�os, EMF y Ecore se han convertido en los est�ndares de facto para la ingenier�a software dirigida por modelos. De esta forma, cada nueva herramienta de modelado que ha ido surgiendo, no s�lo es compatible con Ecore, sino que exige que los modelos que manipula posean un metamodelo definido en Ecore. Por ejemplo, pr�cticamente la totalidad de los lenguajes de transformaci�n de modelos s�lo aceptan como entrada y generan como salida modelos conformes a metamodelos definidos en Ecore. En adelante, nos referiremos a los modelos conformes a metamodelos definidos en Ecore simplemente como modelos Ecore, por simplicidad.  

Dentro de este Proyecto Fin de Carrera, Ecore se ha utilizado para definir un metamodelo para Cassandra, que sirva como representaci�n intermedia del c�digo Cassandra y que facilite su posterior generaci�n.



\section{Epsilon}
\label{sec:back:uml}

%%==================================================================%%
%% Author : Sa�udo Olmedo, Ignacio                                  %%
%%          S�nchez Barreiro, Pablo                                 %%
%% Version: 1.1, 18/06/2014                                         %%
%%                                                                  %%
%% Memoria del Proyecto Fin de Carrera                              %%
%% Background/Epsilon                                               %%
%===================================================================%%

Epsilon~\cite{kolovos:2014} es una familia de lenguajes y herramientas para el desarrollo de software dirigido por modelos. Entre las herramientas de esta familia encontramos lenguajes para realizar transformaciones modelo a modelo, transformaciones modelo a texto, herramientas para definir sintaxis concretas o para chequear la correcci�n sintaxis de un modelo, entre otras funcionalidades.

%%================================================================================%%
%% NOTA((Pablo): Reescrito como arriba y eliminada parte de paja                  %%
%%================================================================================%%
%%
%% \begin{itemize}
%%    \item Herramientas de transformaci�n de modelos;
%%    \ite, validaci�n de modelos o generaci�n de c�digo entre otras 
%%     funcionalidades. 
%% Epsilon es distribuido a trav�s de la plataforma de modelado de lenguajes de 
%% Eclipse.
%% Epsilon proporciona multitud de lenguajes y herramientas para trabajar con modelos. 
%%
%%================================================================================%%

De entre todas las herramientas proporcionadas por Epsilon, este Proyecto Fin de Carrera 
utiliza las siguientes: 
 
\begin{description}
    \item[\emph{EOL} (Epsilon Object Language)] Es el lenguaje b�sico de Epsilon, utiizados por todos sus otros lenguajes, para la manipulaci�n de objetos.
    \item[\emph{ETL} (Epsilon Transformation Language)] Es el lenguaje de transformaci�n modelo a modelo de Epsilon.
    \item[\emph{EGL} (Epsilon Generation Language)] Es el lenguaje de transformaci�n modelo a texto de Epsilon.
    \item[\emph{EUnit}] Es la herramienta proporcionada por Epsilon para la definici�n y ejecuci�n de los casos de prueba que permiten verificar el corrector funcionamiento de las transformaciones desarrolladas.
\end{description}

A continuaci�n describimos con m�s detalle cada uno de estos lenguajes.
    
\subsection{Epsilon Object Language}

\emph{Epsilon Object Language} (EOL) es un lenguaje de programaci�n imperativo utilizado para crear, consultar y modificar modelos Ecore. EOL se puede considerar un lenguaje mezcla de Javascript y OCL, que combina lo mejor de ambos lenguajes. Como tal, proporciona todas las caracter�sticas habituales imperativas que se encuentran en Javascript (por ejemplo, los bucles for y while) y todas las caracter�sticas interesantes de OCL como los filtros sobre colecciones, como por ejemplo Sequence\{1..5\}.select(x \textbar\ x \textgreater\ 3).

%%================================================================================%%
%% NOTA(Pablo): Deber�as usar el del grafo y no uno nuevo, si por cada ejemplo    %%
%%              introduces un nuevo metamdelo, lia bastante                       %%
%%              Puedes poner como ejemplo de funci�n EOL una funci�n que cuente 
%%              el n�mero de nodos azules de un grafo                     
%%              
%%              Adem�s, ponle n�meros de l�nea al c�digo (a mano), y describe 
%%              brevemente qu� hace el c�digo, refiri�ndote a los num�ros de 
%%              l�nea. Por ejemplo, el .all no es tan conocido fuera del mundo 
%%              OCL/modelado
%%================================================================================%%

\begin{figure}[!tb]
  \centering
  \includegraphics[width=.8\linewidth]{background/images/ejCasa.eps} \\
  \caption{Ejemplo metamodelo casa}
  \label{back:fig:ejMetamodeloCasa}
\end{figure}

Para entender mejor el funcionamiento de EOL se expone el siguiente ejemplo. Se ha definido el metamodelo mostrado en la figura~\ref{back:fig:ejMetamodeloCasa}, este meta-modelo consiste en la representaci�n de una casa y las personas que viven en ella. Las personas tienen un nombre y un atributo booleano que representa si una persona est� viva o no. Existe una relaci�n de agregaci�n para reflejar que la casa contiene personas.
Una vez creado el meta-modelo podemos crear un modelo como instancia de ese meta-modelo. Un ejemplo de c�mo funciona EOL puede ser el siguiente: deseamos saber que personas habitan en la casa y est�n vivas. La sintaxis correspondiente ser�a la siguiente (figura~\ref{back:code:codigoEOL}).

\begin{figure}[!tb]
\begin{center}
\begin{footnotesize}
\begin{verbatim}
--------------------------------------------------------
for (person in Person.all){
  if (person.alive == true) {
    person.name.println();
  }
}

//Podemos realizar lo mismo de la siguiente manera:

Person.all.select(r|r.alive==true).name.println();

--------------------------------------------------------
\end{verbatim}
\end{footnotesize}
\end{center}
\caption{Ejemplo c�digo EOL}
\label{back:code:codigoEOL}
\end{figure}


Como vemos la sintaxis es muy similar a cualquier lenguaje orientado a objetos, podemos manipular y consultar los objetos del modelo. Sin embargo EOL no nos permite la definici�n de clases, ya que �stas deben de estar definidas en el metamodelo que EOL manipula.

\subsection{Epsilon Transformation Language}

\emph{Epsilon Transformation Language} (ETL) es un lenguaje de transformaci�n modelo a modelo (\emph{M2M}) basado en reglas. En este proyecto se utiliza ETL para la transformaciones de modelos conceptuales de datos en UML 2.0 en representaciones abstractas de c�digo Cassandra.

%%================================================================================%%
%% NOTA(Pablo): Esto es paja
%%================================================================================%%
%%
%% ETL proporciona las caracter�sticas est�ndar de un lenguaje de transformaci�n, 
%% tambi�n nos permite manipular los modelos de entrada y salida as� como su 
%% c�digo fuente. ETL tiene su propia sintaxis sin embargo utiliza el lenguaje 
%% EOL como base.
%%
%%================================================================================%%

%%================================================================================%%
%% NOTA(Pablo): Este ejemplo p�salo al cap�tulo 1 y en esta secci�n simplemente 
%%              dices que un ejemplo de etl se puede encontrar en el cap�tulo 1
%%              Ponle n�meros a las l�neas de c�digo y describe un poco el c�digo 
%%              m�s o menos l�nea a l�nea sin ponerte pesado, tal como lo
%%              tienes hecho est� bien      
%%================================================================================%%

Recordando el ejemplo de la Red de computadores y el Grafo detallado en el capitulo anterior (secci�n 1.2). Deseamos realizar la transformaci�n de un modelo de tipo grafo a un modelo de tipo red para ello definiremos una serie de reglas de transformaci�n utilizando para ello ETL. En primer lugar necesitamos definir el modelo del grafo para poder realizar la transformaci�n a un modelo de tipo Red.
La figura~\ref{back:code:codigoETL} muestra el c�digo que realiza el proceso de transformaci�n de un Grafo a una Red.

\begin{figure}[!tb]
\begin{center}
\begin{footnotesize}
\begin{verbatim}

rule Arista2Cable
transform a : Grafo!Arista
to r : Red!Cable {	
    r.nameCable = a.nombreArista;
    if (a.parent.isDefined()) {
        r.parent=Red;
        for (nodoArista in a.children) {
            if(nodoArista.color=TColor#R){
                var PC : new Red!PC;
                PC.nameNodo="PC"+iPC;
                r.children.add(PC);
                iPC=iPC+1;	
            }
            else{
                var Router : new Red!Router;
                Router.nameNodo="Router"+iRouter;
                r.children.add(Router);
                iRouter=iRouter+1;
            }
        }
    }	
}

\end{verbatim}
\end{footnotesize}
\end{center}
\caption{Ejemplo c�digo ETL}
\label{back:code:codigoETL}
\end{figure}

En este c�digo encontramos solo una regla. Esta regla transforma aristas del grafo a cables de la red. La primera instrucci�n copia el nombre de la arista al cable. A continuaci�n la primera condicional cuestiona si esa arista tiene un padre definido en caso afirmativo asigna el cable a la red. A continuaci�n por cada nodo se crea o bien un PC o un router dependiendo del color del nodo que se est� analizando (rojo-PC, azul-Router). En siguiente lugar se asigna el nodo creado al cable correspondiente y finalmente se a�ade a la red. Este proceso se repite por cada arista del grafo. Una vez ejecutado este c�digo dado un modelo de entrada de tipo Grafo obtenemos un modelo equivalente de tipo Red que cumple las reglas definidas en el meta-modelo.

\subsection{Epsilon Generation Language}

\emph{Epsilon Generation Language} (EGL)~\cite{louis:2008} es un lenguaje utilizado para la transformaci�n de modelos a texto (\emph{M2T}) basado en plantillas. 

EGL puede utilizarse para transformar modelos en cualquier tipo de lenguaje, por ejemplo c�digo ejecutable Java, c�digo HTML o incluso aplicaciones completas que comprenden el c�digo en varios lenguajes (por ejemplo, HTML, Javascript y CSS). En este proyecto se utiliza EGL para la generaci�n de c�digo Cassandra Query Language (CQL) a partir de modelos UML 2.0.

%%================================================================================%%
%%
%% NOTA(Pablo): Este ejemplo p�salo al cap�tulo 1 y en esta secci�n simplemente
%%              dices que un ejemplo de egl se puede encontrar en el cap�tulo 1
%%              Ponle n�meros a las l�neas de c�digo y describe un poco el c�digo
%%              m�s o menos l�nea a l�nea sin ponerte pesado. 
%%              Esto deber�as describirlo en un poco m�s en detalle
%%
%%================================================================================%%

Cada plantilla de EGL contiene varias secciones. Cada secci�n puede ser est�tica o bien din�mica. Una secci�n est�tica contiene texto que aparecer� directamente y tal como est� en la salida generada por la plantilla. Una secci�n din�mica comienza con la secuencia '[\%' y termina con la secuencia '\%]'. La secci�n din�mica contiene c�digo lenguaje EOL.

La figura~\ref{back:code:codigoEGL} muestra como se realiza la generaci�n de c�digo HTML utilizando para ello el modelo generado de una red a partir de un grafo (ver secci�n anterior).

\begin{figure}[!tb]
\begin{center}
\begin{footnotesize}
\begin{verbatim}
--------------------------------------------------------
[%
    var red: Red := Red.allInstances().at(0);
%]

<html>
    <head>
        <title> Red </title>
    </head>
    <body>
        <h1>Conexiones</h1>				
        <table  border="1">
            <col style="width: 200px" />
            <col style="width: 100px" span="3" />
            [% for (conexiones in red.conexiones){%]
            <tr>
                <th scope="row">[%=conexiones.nameCable%]</th>
                [% for (nodos in conexiones.children){%]
                    <td>[%=nodos.nameNodo%]</td>
                [%	}%]
            </tr>
            [%	}%]
        </table>
    </body>
</html>
--------------------------------------------------------
\end{verbatim}
\end{footnotesize}
\end{center}
\caption{Ejemplo c�digo EGL}
\label{back:code:codigoEGL}
\end{figure}

%%================================================================================%%
%% NOTA(Pablo): Esto sobra
%%================================================================================%%
%%
%% Como vemos en el c�digo la integraci�n del c�digo EGL junto con HTML es total, 
%% en este sencillo c�digo se genera una p�gina HTML con una tabla que muestra 
%% varias filas, una fila por cada conexi�n entre dos componentes de la red. 
%%
%%================================================================================%%

%%================================================================================%%
%% NOTA(Pablo): Debes a�adir una secci�n sobre EUnit                              %%
%%================================================================================%%

\section{Cassandra}
\label{sec:back:tente}

%%==================================================================%%
%% Author : Sa�udo Olmedo, Ignacio                                  %%
%%          S�nchez Barreiro, Pablo                                 %%
%% Version: 1.1, 18/06/2014                                         %%
%%                                                                  %%
%% Memoria del Proyecto Fin de Carrera                              %%
%% Background/Cassandra                                             %%
%===================================================================%%
%http://www.nosql-database.org/
%[chalmers]
%http://www.acens.com/wp-content/images/2014/02/bbdd-nosql-wp-acens.pdf
%http://www.strozzi.it/cgi-bin/CSA/tw7/I/en_US/NoSQL/Philosophy%20of%20NoSQL

Esta secci�n realiza una breve descripci�n sobre \imp{Cassandra}~\cite{lith:2010}, un sistema gestor de datos NoSQL basado en columnas. Describiremos brevemente su modelo de datos y la sintaxis de su lenguaje de gesti�n de datos. 

%% Por �ltimo, justificaremos el por qu� de la elecci�n de Cassandra dentro de este Proyecto Fin de Carrera.

%http://books.google.es/books?id=tv5iO9MnObUC&printsec=frontcover&dq=nosql+books&hl=es&sa=X&ei=XHjrU7mWFpGg7Abk6oCIDw&ved=0CDoQ6AEwAA#v=onepage&q=nosql%20books&f=false
%%=====================================================================%%
%% NOTA(Pablo): Esto es paja y se contradice en parte con lo que est�  %%
%%              escrito en la introducci�n, aparte de no aportar nada  %%
%%              al proyecto en s�                                      %%
%%=====================================================================%%
%%
%% Desde que naci� \imp{SQL} en el a�o 1974, �ste  se ha convertido en el lenguaje de
%% consultas utilizado por excelencia. En los �ltimos a�os ha surgido otra vertiente
%% denominada NoSQL, esta vertiente surge por la necesidad de manejo de grandes
%% vol�menes de informaci�n no estructurada, distribuida y accedida con la mayor rapidez
%% posible.
%%
%% NoSQL es literalmente (como es evidente) la combinaci�n de dos palabras: No y SQL,
%% hoy en d�a este t�rmino se utiliza para denominar todas las bases de datos que no
%% siguen los principios de los sistema de gesti�n de bases de datos relacionales
%% tradicionales. Es usado en plataformas como Facebook o Twitter. Podemos encontrar
%% varios tipos de bases de datos no relacionales, por ejemplo: Bases de datos
%% documentales, bases de datos orientadas a columnas, orientadas a grafos ente otros.
%%
%%=====================================================================%%
Cassandra es un sistema gestor de datos NoSQL. En general, existen varias diferencias entre estos sistemas NoSQL y los tradicionales sistemas SQL (\emph{Structured Query Language})~\cite{}, que son:

\begin{enumerate}
    \item Los sistemas NoSQL no garantizan las propiedades \emph{ACID} (\emph{Atomicity, Consistency, Isolation y Durability}). Por ejemplo, en ciertos casos, Cassandra no garantiza la integridad de los datos, pudiendo contener datos incongruentes. Por ejemplo, una asignatura podr�a aparecer en la lista de materias cursadas por un alumno, pero dicho alumno podr�a no aparecer en la lista de alumnos de esa asignatura, lo que ser�a no consistente. Esto se debe a que sistemas como Cassandra introducen cierta redundancia para favorecer ciertos tipos de consultas.
    \item Los sistemas NoSQL no utilizan SQL como lenguaje de consultas (de ah� su nombre). Algunos bases de datos NoSQL utilizan SQL como lenguaje de apoyo. Sin embargo, la mayor�a utilizan su propio lenguaje de gesti�n de datos. Por ejemplo, Cassandra  utiliza CQL (\emph{Cassandra Query Language})~\cite{}.
    \item Ciertos operadores de los sistemas relacionales, como los \emph{joins}, suelen eliminarse de estos sistemas, ya que al manejar grandes vol�menes de informaci�n pueden llegar a sobrecargar el sistema.
    \item Escalan horizontalmente y trabajan de manera distribuida. De esta forma, se puede aumentar la capacidad o rendimiento del sistema simplemente a�adiendo nuevos nodos.
\end{enumerate}

%%=====================================================================%%
%% NOTA(Pablo): Esto es paja                                           %%
%%=====================================================================%%
%%
%% Cassandra fue dise�ado por Avinash Lakshman (uno de los creadores
%% de Amazon's Dynamo) y Prashant Malik (Ingeniero de Facebook), en
%% estos momentos se encuentra en producci�n para Facebook.
%%
%%=====================================================================%%

Dentro del mundo de las bases de datos no relaciones Cassandra es considerada
un h�brido de los sistemas \emph{orientados a columnas} y las bases de datos \emph{basadas en clave-valor}. Los basados en columnas tienen como peculiaridad que almacenan los datos en forma de columna. Esto permite el acceso a los datos de forma r�pida, b�sicamente, utilizan una tabla hash en la que existe una clave �nica y un puntero a un elemento de datos en particular. Los sistemas basados en clave-valor son aquellos que asocian los valores a una determinada clave esto permite la recuperaci�n y escritura de informaci�n de forma muy r�pida y eficiente.

%%=====================================================================%%
%% NOTA(Pablo): Hay que que poner una imagen que ilustre claramente
%%              la diferencia entre ambos casos
%%=====================================================================%%

%%=====================================================================%%
%% NOTA(Pablo): Esto es paja                                           %%
%%=====================================================================%%
%%
%% Cassandra re�ne las  tecnolog�as de sistemas distribuidos de
%% Amazon Dynamo y el modelo de datos BigTable de Google. Por ejemplo,
%% al igual que Dynamo, es consistente. Como BigTable, proporciona un
%% modelo de datos basado en column families siendo considerado el
%% sistema orientado a columnas m�s popular.
%%
%%=====================================================================%%

%https://cassandra.apache.org/ http://www.datastax.com/documentation/cql/3.0/cql/ddl/ddl_intro_c.html

%%=====================================================================%%
%% NOTA(Pablo): Aqu� hay que poner una imagen que describa exactamente %%
%%              c�mo es el modelo de datos de Cassandra, y luego       %%
%%              explicar cada elemento con relaci�n al modelo          %%
%%              Decribir el modelo sin hacer referencia a las bases de %%
%%              datos relacionales.                                    %%
%%=====================================================================%%

A continuaci�n se explican brevemente el modelo de datos en el cual se basa Cassandra.
En Cassandra un \imp{keyspace} es el equivalente a un schema en los sistemas de bases de datos relacionales.
El conocido t�rmino de tabla en las bases de datos relacionales tiene su equivalente en Cassandra llamado \imp{column family}, las column families se guardan en ficheros separados y son ordenadas por su key.
Una \imp{columna} es la unidad de almacenamiento b�sica, est� formada de tres campos: Nombre, valor y timestamp. El nombre y el valor se almacena como una matriz de bytes sin procesar y pueden ser de cualquier tama�o. Los tres valores anteriores son introducidos por el cliente, incluido el timestamp. Un ejemplo de la estructura de una column:

%%=====================================================================%%
%% NOTA(Pablo): Esta imagen hay que sustituirla por otra que se vea    %%
%%              m�s clara, aunque si haces una en la cual se vea el    %%
%%              modelo de datos de Cassandra, la                       %%
%%=====================================================================%%


\begin{table}[!hbt]
\begin{center}
\begin{tabular}{||l | c | r||}
\hline
\hline
\textbf{Nombre de la columna} & Username \\
\hline
\textbf{Valor de Username} & Ignacio \\
\hline
\textbf{Timestamp} & 123456789 \\
\hline
\end{tabular}
\caption{Ejemplo estructura columna}
\end{center}
\end{table}
Por lo tanto un keyspace puede contener varias column families y una column family a su vez contiene varias columnas. La estructura de una column family queda como se puede observar en la figura~\ref{back:fig:estructuraCF}.
\begin{figure}[!tb]
  \centering
  \includegraphics[width=.8\linewidth]{background/images/estructuraCF.eps} \\
  \caption{Estructura column family}
  \label{back:fig:estructuraCF}
\end{figure}

	
En cuanto a la sintaxis del lenguaje \imp{CQL} es muy similar a \imp{SQL}, CQL contiene sintaxis ya conocida de SQL como \imp{INSERT}, \imp{DELETE} o \imp{UPDATE}.

La Figura~\ref{back:code:codigoCQL} refleja un ejemplo breve de c�digo CQL ejecutable.

%%=====================================================================%%
%% NOTA(Pablo): Igual que antes, ponle n�meros al c�digo y luego       %%
%%              lo comentas                                            %%
%%=====================================================================%%


\begin{figure}[!tb]
\begin{center}
\begin{footnotesize}
\begin{verbatim}

DROP KEYSPACE twitter;

CREATE KEYSPACE twitter
WITH replication = {'class':'SimpleStrategy', 'replication_factor':2};

USE twitter;

CREATE TABLE Tweet(
       UUID text,
       usernameTw text,
       body text,
       PRIMARY KEY(UUID)
);

CREATE TABLE User(
       username text,
       password text,
       followers set<text>,
       followings set<text>,
       tweets_written list<text>,
       PRIMARY KEY(username)
);

\end{verbatim}
\end{footnotesize}
\end{center}
\caption{Ejemplo c�digo CQL}
\label{back:code:codigoCQL}
\end{figure}

%%=====================================================================%%
%% NOTA(Pablo): En realidad esos conceptos deber�as coment�rlos aqu�   %%
%%=====================================================================%%
%%
%% Conceptos relacionados con Cassandra son ampliados en el siguiente 
%% capitulo a la hora de definir el meta-modelo de Cassandra.
%%
%%=====================================================================%%

%%=====================================================================%%
%% NOTA(Pablo): Estar�a bien poner algo de esto, pero si dice algo se  %%
%%              debe justificar. Sino, mejor no ponerlo                %% 
%%=====================================================================%%
%% 
%% Las bases de datos basadas en Cassandra se suelen utilizar cuando es 
%% necesaria proporciona una buena escalabilidad y alta disponibilidad 
%% sin comprometer el rendimiento. Adem�s proporciona gran estabilidad 
%% en cuanto a la replicaci�n de datos a trav�s de m�ltiples centros de 
%% datos, consiguiendo una menor latencia para sus usuarios 
%% y la tranquilidad de que no existan perdidas de datos ante ca�das.
%%
%%=====================================================================%%

%%=====================================================================%%
%% NOTA(Pablo): En realidad esos conceptos deber�as coment�rlos aqu�   %%
%%=====================================================================%%
%%
%% Cassandra utiliza el lenguaje de consultas \imp{CQL (Cassandra Query Language)} con una sintaxis muy similar 
%% a SQL aunque mas limitado que este. Como herramienta de manipulaci�n y administraci�n de datos se utiliza 
%% \imp{DataStax OpsCenter} que ofrece una interfaz de usuario basada en navegador que sirve para la gesti�n 
%% y monitorizaci�n de los cluster Cassandra en una �nica terminal de administraci�n.
%%
%%=====================================================================%%

\section{Planificaci�n}
\label{sec:back:slicer}

%%==================================================================%%
%% Author : Sa�udo Olmedo, Ignacio                                  %%
%%          S�nchez Barreiro, Pablo                                 %%
%% Version: 1.2, 18/06/2013                                         %%
%%                                                                  %%
%% Memoria del Proyecto Fin de Carrera                              %%
%% Background/Planificacion                                         %%
%===================================================================%%

El objetivo de este proyecto de fin de carrera es la implementaci�n de un generador de c�digo Cassandra a partir de modelos UML. El proceso de desarrollo as� como el de aprendizaje que se ha seguido para la realizaci�n del proyecto es descrito a continuaci�n.

La primera tarea como es evidente consisti� en adquirir los conocimientos necesarios para el desarrollo del proyecto. En primer lugar todo lo relacionado con el proceso de modelado de un lenguaje y transformaci�n de lenguajes (\cite{kleppe:2008}). Tambi�n fueron necesarios conocimientos sobre la Ingenier�a y el Desarrollo Dirigido por Modelos, as� como de la sintaxis, arquitectura y funcionamiento de Cassandra.

A continuaci�n se comenz� a trabajar con la herramienta Epsilon, sus lenguajes EOL, EGL, ETL y finalmente EUnit como herramienta para las pruebas de los modelos y c�digo generados. Adem�s el lenguaje para la definici�n de lenguajes de modelado Eclipse Modeling Framework (EMF).
Para conocer c�mo funcionaban estos lenguajes se desarrollaron una serie de casos pr�cticos para familiarizarse con los m�todos de transformaci�n as� como con la herramienta, para ello se realizo el proceso completo para crear un generador de c�digo desde la transformaci�n entre modelos hasta la transformaci�n modelo-c�digo. Estos casos de prueba son los que se exponen en la secci�n de Epsilon (secci�n 2.3).

Una vez conocido estos conceptos se estudiaron las reglas de transformaci�n a aplicar para transformar un modelo UML a un modelo Cassandra, estas reglas fueron propuestas por \cite{pablo:2013}.

Tras estas tareas de adquisici�n de conocimientos se comenz� a trabajar en el generador de c�digo Cassandra empezando por la transformaci�n de modelos UML a modelos Cassandra. Estas transformaciones son expuestas en el siguiente cap�tulo. Una vez finalizada la transformaci�n se comenz� a trabajar en el generador de c�digo Cassandra, esta tarea es descrita en el cap�tulo 5. Tras realizar dicha tarea se realizaron una serie de casos de prueba para verificar si los resultados que otorgaba el generador de c�digo eran los esperados, para esta tarea utilizamos la herramienta EUnit.
Finalmente y tras generar varios casos de ejemplo se instalo DataStax OpsCenter y Apache Cassandra, se ejecutaron los resultados generados y se comprob� su correcto funcionamiento. 

\section{Sumario}
\label{sec:back:sumario}

%%==================================================================%%
%% Author : Sa�udo Olmedo, Ignacio                                  %%
%% Author : S�nchez Barreiro, Pablo                                 %%
%% Version: 1.5, 15/05/2014                                         %%
%%                                                                  %%
%% Memoria del Proyecto Fin de Carrera                              %%
%% m2t/Sumario                                                      %%
%%==================================================================%%

Durante este cap�tulo se ha descrito el proceso de desarrollo del generador de c�digo. En primer lugar se ha presentado el c�digo del generador de c�digo y parte de la sintaxis utilizada para construirlo as� como la descripci�n de su realizaci�n. A continuaci�n se ha descrito el proceso de transformaci�n del caso de estudio introducido en el cap�tulo dos. Finalmente se han presentado las pruebas realizadas as� comola herramienta utilizada.   
