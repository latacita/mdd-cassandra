%%==================================================================%%
%% Author : Sa�udo Olmedo, Ignacio                                  %%
%% Author : S�nchez Barreiro, Pablo                                 %%
%% Version: 1.5, 15/05/2014                                         %%
%%                                                                  %%
%% Memoria del Proyecto Fin de Carrera                              %%
%% m2t/Sumario                                                      %%
%%==================================================================%%

Este proyecto de fin de carrera ha cubierto todos los objetivos planteados desde el principio, es posible que el proyecto deje alg�n aspecto por completar para que el trabajo presentado tenga la apariencia de un producto profesional sin embargo hay propiedades en las que la herramienta Epsilon ha puesto muchas dificultades, por ejemplo: muchos bugs de car�cter desconocido, dificultades para el empaquetado del generador de c�digo, funcionalidades sin cubrir por la herramienta EUnit..
Los problemas sin cubrir son comprensibles ya que al ser una herramienta desarrollada por terceros estos defectos est�n fuera de nuestro alcance por lo que dif�cilmente pueden ser subsanados. Adem�s hay que tener en cuenta que Epsilon es una herramienta a�n joven.
Otro posible futura funcionalidad encontrada es que Epsilon ofrezca facilidades para el empaquetado del generador de c�digo de manera que se pueda crear un plug-in que se integre en Eclipse, de esta manera cualquier usuario de Eclipse con la instalaci�n sencilla del plug-in podr�a ejecutar este generador de c�digo de manera muy sencilla.
Otra de las tareas que se podr�a desarrollar es la integraci�n del generador de c�digo con Apache Cassandra esto podr�a resultar �til a los desarrolladores ya que con el dise�o UML del sistema a implementar se podr�a generar el repositorio de datos de manera muy sencilla.
