%%==================================================================%%
%% Author : Abascal Fern�ndez, Patricia                             %%
%%          S�nchez Barreiro, Pablo                                 %%
%% Version: 1.3, 18/06/2013                                         %%                                                                                    %%                                                                  %%
%% Memoria del Proyecto Fin de Carrera                              %%
%% Archivo ra�z                                                     %%
%%==================================================================%%

\cdpchapter{Resumen}

Ciertas aplicaciones de alta demanda, accesibles a trav�s de internet, como Twitter o Amazon, poseen requisitos muy particulares que resultan complejos de satisfacer utilizando los tradicionales sistemas gestores de bases de datos relacionales. Para resolver este problema, han ido apareciendo en los �ltimos a�os una serie de tecnolog�as de almacenamiento y recuperaci�n de datos conocidas como NoSQL. No obstante, dichas tecnolog�as han aparecido a nivel de implementaci�n, no siendo posible a�n construir sistemas no relacionales desde modelos conceptuales de alto nivel, tal como se ha venido realizando para el caso relacional desde hace d�cadas.

El objetivo de este proyecto es crear haciendo uso de las modernas tecnolog�as de desarrollo software dirigido por modelos, una herramienta que permita transformar un modelo de datos conceptual de alto nivel expresado en UML 2.0 en una implementaci�n para un sistema de almacenamiento de datos NoSQL. Concretamente se utilizar� el sistema de datos basado en columnas llamado Cassandra.

\paragraph{Palabras Clave} \ \\

Desarrollo Dirigido por Modelos, Ingenier�a Dirigida por Modelos, Generaci�n de C�digo, Cassandra, Epsilon, UML, CQL.



