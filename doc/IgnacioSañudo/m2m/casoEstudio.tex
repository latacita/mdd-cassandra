%%==========================================================================%%
%% Author : Sa�udo Olmedo, Ignacio                                          %%
%% Author : S�nchez Barreiro, Pablo                                         %%
%% Version: 1.2, 21/04/2014                                                 %%
%%                                                                          %%
%% Memoria del Proyecto Fin de Carrera                                      %%
%% M2M/Caso de estudio                                                      %%
%%==========================================================================%%

Como se explicaba en el cap�tulo 2 (secci�n 2.1), el objetivo consiste en la creaci�n de un generador de c�digo de una versi�n simplificada de Twitter llamada Twissandra.
En esta secci�n se reproducir�n los procesos M2M y M2T en el siguiente cap�tulo, todo esto bajo el proceso de desarrollo dirigido por modelos. Esta secci�n est� dedicada a describir la transformaci�n del modelo UML de Twissandra a un modelo Cassandra, para ello partimos del modelo UML de la figura~\ref{back:fig:twissandra}. Una vez establecidas las reglas de transformaci�n entre modelos podemos realizar la generaci�n del c�digo aunque esto se analizar� en el siguiente cap�tulo.

En primer lugar, por cada paquete estereotipado como <<\imp{dataModel}>>, se crea un nuevo keyspace. El nombre del keyspace ser� el nombre que se ha definido en el modelo de datos UML. Los atributos restantes de las meta-clases del keyspace se establecen en sus valores definidos por defecto en el modelo UML. A continuaci�n, todos los elementos correspondientes de ese paquete se procesan.

A continuaci�n se realiza un marcado del atributo username de la clase User como clave, ya que el par�metro \imp{isID} del atributo username en el modelo UML est� marcado como \imp{true}. En el caso de las clases \imp{FollowingRelationship} y la clase \imp{Tweet} al no tener un atributo marcado como clave generamos dos columnas clave autom�ticamente para cada clase, llamadas \imp{FollowingRelationship\_id} y \imp{tweet\_id} respectivamente.
La clase \imp{User} del modelo UML es transformada en una column family llamada \imp{User}. Una vez procesadas las clases UML y transformarlas a su correspondiente column family se proceden a transformar los atributos y asociaciones. De manera similar para aquellos atributos del modelo UML cuya multiplicidad sea igual a uno se realiza una transformaci�n simple, por ejemplo el atributo \imp{username} y \imp{password} se transforman en dos columnas Cassandra, ambas del tipo \imp{text}. Estas columnas est�n contenidas en la column family \imp{User}. De la misma forma se transforman los atributos \imp{body} y \imp{time} de la clase \imp{Tweet} y el atributo \imp{timestamp} de la clase \imp{FollowingRelationship}.
En el caso del atributo del modelo UML \imp{email} cuya multiplicidad es mayor de uno y tiene las propiedades \imp{isUnique} establecida en \imp{false} y la propiedad \imp{isOrdered} establecida en \imp{false} (en el modelo no se puede apreciar pero est� configurado as� en el modelo UML), se transforma este atributo en una colecci�n de tipo \imp{set} llamada \imp{email} cuyo tipo primitivo ser� \imp{text}, esta colecci�n estar� dentro de la column family \imp{User}.

En cuanto a las asociaciaciones recordamos que tenemos dos tipos, las de multiplicidad igual a uno y las de multiplicidad mayor de uno. Para la asociaci�n de la clase \imp{User} cuya multiplicidad es igual uno, se crea una nueva columna llamada \imp{user\_username} (recordemos que username es la clave de la column family user) y esta columna es a�adida a la column family \imp{Tweet}. Para las asociaciones de multiplicidad mayor de uno, por ejemplo la asociaci�n llamada \imp{userline} se crea una dynamic column family llamada \imp{User\_userline}. A continuaci�n una columna llamada \imp{user\_username} de tipo text es a�adida a esta column family. Despu�s una columna llamada \imp{tweet\_id} de tipo \imp{uuid} es a�adida (el atributo \imp{tweet\_id} fue creado en la column family \imp{tweet} al no tener clave). Las columnas \imp{user\_username} y \imp{tweet\_id} son designadas como primary key, la columna \imp{user\_username} ser� la partition key y la columna \imp{tweet\_id} ser� la cluster key.


