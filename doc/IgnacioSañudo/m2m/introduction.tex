%%==================================================================%%
%% Author : Sa�udo Olmedo, Ignacio                                  %%
%% Author : S�nchez Barreiro, Pablo                                 %%
%% Version: 1.4, 21/06/2014                                         %%
%%                                                                  %%
%% Memoria del Proyecto Fin de Carrera                              %%
%% M2M/Introduccion                                                 %%
%%==================================================================%%


El primer paso a la hora de desarrollar un generador de c�digo es establecer una serie de reglas de correspondencia entre el modelo de entrada y el modelo de salida. Estas reglas han de contemplar los distintos tipos de elementos que pueden aparecer en ambos lenguajes, as� como elementos que no aparecen en uno de los dos lenguajes y tiene importancia en el otro.
En este caso, consiste en establecer reglas de correspondencia entre elementos UML 2.0 y el lenguaje Cassandra Query Language (CQL).
Para la definici�n de estas reglas se ha utilizado el lenguaje Epsilon Transformation Language (ETL) y el lenguaje Epsilon Object Language (EOL).
Tras implementar estas reglas y comprobar que son funcionales se puede implementar el generador de c�digo pero este tema se trata en el siguiente capitulo.
En cuanto a la construcci�n EMF
El meta-modelo utilizado de UML es el que proporciona la herramienta EMF por defecto.