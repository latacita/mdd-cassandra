%%==================================================================%%
%% Author : Sa�udo Olmedo, Ignacio                                  %%
%% Author : S�nchez Barreiro, Pablo                                 %%
%% Version: 1.5, 15/05/2014                                         %%
%%                                                                  %%
%% Memoria del Proyecto Fin de Carrera                              %%
%% m2t/Sumario                                                      %%
%%==================================================================%%

Esta secci�n analiza el proceso de creaci�n del generador de c�digo. Para crear este generador de c�digo se utiliza el lenguaje Epsilon Generation Language (EGL), como se comentaba en el capitulo 2 es un lenguaje utilizado para la generaci�n de c�digo basado en la transformaci�n de modelos, el lenguaje a generar es Cassandra Query Language (CQL) este lenguaje de consultas es muy similar a SQL, sin embargo existen peque�as diferencias que se han citado a lo largo del proyecto por ejemplo: la definici�n de claves, tipos de dato etc.
En la figura~\ref{back:code:m2t} se muestra parte del c�digo del generador de c�digo. Esta secci�n de c�digo muestra c�digo EGL. En primer lugar se realiza la definici�n del keyspace para ello hay que definir el nombre del keyspace que parte del modelo transformado Cassandra y la estrategia de replicaci�n (ver capitulo 3, secci�n 2).
A continuaci�n se realiza la creaci�n de una column family en CQL. En primer lugar se definen las columnas de la tabla. Por cada columna de la column family se obtiene el nombre de la columna y se define el tipo de dato (primitivo,map,set/list) junto con su nombre. El resto de c�digo se ha omitido por razones de espacio sin embargo se detalla a continuaci�n.
Tras la definici�n de la columna se define la primary key. Para la definici�n de la primary key hay que diferenciar si la column family es din�mica o est�tica. En caso de ser est�tica la definici�n ser�a la cl�sica: PRIMARY KEY(ColumnaClave).
En el caso de tratarse de una column family din�mica la construcci�n modelo-c�digo es distinta, la m�s frecuente es la siguiente: PRIMARY KEY(partitioning key, clustering key\_1 ... clustering key\_n)
Sin embargo hay que tener en cuenta que en la construcci�n tambi�n es posible tener una partition key compuesta, es decir, una partition key formada por varias columnas, para ello se utilizan par�ntesis y as� delimitamos el conjunto de partici�n. Quedando una posible definici�n de la primary key de la siguiente manera: PRIMARY KEY((partitioning key\_1, ... partitioning key\_n), clustering key\_1 ... clustering key\_n). Este proceso se repite por cada column family del modelo.

\begin{figure}[!tb]
\begin{center}
\begin{footnotesize}
\begin{verbatim}
--------------------------------------------------------
DROP KEYSPACE [%=keyspace.name%];
CREATE KEYSPACE [%=keyspace.name%]
WITH replication = {'class':'[%=keyspace.replicaPlacementStrategy%]', 'replication_factor':[%=keyspace.replicationFactor%]};

USE [%=keyspace.name%];

[% for (cf in keyspace.columnFamilies){ tableKey="";%]
CREATE TABLE [%=cf.name%](
	[% for (cols in cf.columns){ //este for define cada grupo de columnas
		s=cols.name.toString();
		for (c in cols.type){
		
			if (c.isTypeOf(PrimitiveType))  //definicion del tipo primitivo
				s=s+" "+c.kind.toString();
				
			if (c.isTypeOf(MapType)) //definicion del tipo map
				s=s+" Map"+"<"+c.keyType.toString()+","+c.baseType.toString()+">";
				
 			if(c.isTypeOf(CollectionType)) //definicion del tipo set o list
				s=s+" "+c.kind.toString()+"<"+c.keyType.toString()+">";
			
		}
);
[%}%]
--------------------------------------------------------
\end{verbatim}
\end{footnotesize}
\end{center}
\caption{Regla de transformaci�n XX}
\label{back:code:m2t}
\end{figure} 