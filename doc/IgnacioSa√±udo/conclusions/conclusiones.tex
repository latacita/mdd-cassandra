%%==================================================================%%
%% Author : Abascal Fern�ndez, Patricia                             %%
%% Author : S�nchez Barreiro, Pablo                                 %%
%% Version: 1.5, 15/05/2013                                         %%
%%                                                                  %%
%% Memoria del Proyecto Fin de Carrera                              %%
%% Conclusiones, conclusiones                                       %%
%%==================================================================%%

Para el desarrollo de estos generadores de c�digo se ha utilizado la herramienta Epsilon que ofrece funcionalidades bastante potentes, pero dada su reciente aparici�n y su constante desarrollo, a�n presentan ciertas carencias que han de ser resueltas. Siendo una de esas carencias que m�s necesaria e imprescindible considero, la integraci�n con Eclipse. Ha sido imposible empaquetarlo como plug-in debido a una excepci�n interna producida por el llenado memoria durante la ejecuci�n del mismo. Otra carencia que debe mejorar es la incompatibilidad de dicho proceso con la librer�a \emph{Java.swt} para la generaci�n del entorno gr�fico de nuestros generadores de c�digo.

En muchos momentos del desarrollo nos encontramos con errores que los propios desarrolladores desconoc�an, por ejemplo a la hora de tratar modelos UML que tuvieran clases enumeradas se lanzaba una excepci�n que, tras reportarlo como bug y varios meses despu�s, fue solucionado por los desarrolladores en la �ltima versi�n de la herramienta. Incluso durante la fase de pruebas con \emph{EUnit} me top� con ciertas funcionalidades que no hab�an sido implementadas y que necesitaba para la comparaci�n de generaci�n de c�digo obtenido con la generaci�n de c�digo esperado,  por lo que tuve que escribir varias veces en el foro oficial de Epsilon y tratar, una vez m�s, con los desarrolladores que pusieron soluci�n a dichos imprevistos.

El mayor freno que sufri� el desarrollo fue debido en la mayor�a de los casos a problemas con las propias herramientas. A�n as�, fue muy satisfactorio poder desarrollar los generadores de c�digo utilizando un lenguaje funcional, ya que difiere de lo que en muchas ocasiones aprend� a los lenguajes aprendidos a lo largo de mis estudios. Creo que la construcci�n de Software Orientado a Caracter�sticas y las L�neas de Producto Software tienen mucho futuro, ya que permiten ahorrar tiempo, costes y crear productos acordes a las necesidades de cada usuario concreto y en cuanto las herramientas consigan estandarizarse no me cabe duda que ser�n utilizadas de un modo mucho m�s frecuente en el futuro. 