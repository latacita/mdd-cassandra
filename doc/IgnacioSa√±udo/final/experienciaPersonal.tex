%%==================================================================%%
%% Author : Sa�udo Olmedo, Ignacio                                  %%
%% Author : S�nchez Barreiro, Pablo                                 %%
%% Version: 1.5, 15/05/2014                                         %%
%%                                                                  %%
%% Memoria del Proyecto Fin de Carrera                              %%
%% m2t/Sumario                                                      %%
%%==================================================================%%

Esta secci�n describe las experiencias personales acontecidas a lo largo del proyecto.

Las tareas de aprendizaje de lenguajes como EOL, ETL o EGL no resultaron muy costosas sin embargo siempre cuesta adentrarse en materias desconocidas ya que a pesar de que EOL por ejemplo es similar a OCL (Object Constraint Language) son conceptos que no se ven mucho durante la carrera.

En cuanto a la tarea de desarrollo del generador de c�digo se utiliz� la herramienta Epsilon que ofrece funcionalidades que facilitan el trabajo de crear generadores de c�digo, sin embargo el mayor problema de estas herramientas es que al ser de reciente creaci�n el numero de errores que surgen durante el desarrollo son molestos y numerosos, por ejemplo existen errores sin identificar durante la ejecuci�n de los casos de prueba. Tambi�n surgen dificultades a la hora de poner en marcha el generador de c�digo ya que algunos par�metros de configuraci�n de Epsilon no son descritos en los manuales, a pesar de que Epsilon proporciona a los usuarios manuales y tutoriales completos y did�cticos pero a nivel elemental.
A pesar de algunos infortunios o problemas a lo largo del desarrollo el resultado del trabajo es satisfactorio.

El paradigma del Desarrollo Dirigido por Modelos en mi opini�n ser� una materia b�sica a la hora de estudiar tecnolog�as y herramientas que pueden facilitar la vida a las empresas. Personalmente creo que estas tecnolog�as utilizadas para la automatizaci�n de construcci�n software son parte del futuro y ser�n materias troncales para futuros alumnos que estudien Ingenier�a Inform�tica, ya que permiten a las empresas reducci�n de costes, ahorro de tiempo y reutilizaci�n de componentes tres pilares clave en materia de optimizaci�n de los recursos de una empresa software. 