%%==================================================================%%
%% Author : Sa�udo Olmedo, Ignacio                                  %%
%%          S�nchez Barreiro, Pablo                                 %%
%% Version: 1.4, 18/06/2013                                         %%                                                                                    %%                                                                  %%
%% Memoria del Proyecto Fin de Carrera                              %%
%% Introduccion/Motivaci�n y objetivos                              %%
%%==================================================================%%



Como hemos visto a lo largo de la introducci�n el enfoque que proporciona el Desarrollo Dirigido por Modelos as� como la Ingenier�a Dirigida por Modelos nos proporciona m�ltiples ventajas respecto al desarrollo tradicional por lo que la utilizaci�n de este enfoque es apropiada para la realizaci�n del generador de c�digo as� como para las transformaciones entre modelos.

La utilizaci�n de modelos UML respecto a modelos escritos en Cassandra a la hora de especificar una base de datos no relacional proporciona una abstracci�n para aquellos usuarios que no est�n muy familiarizados con el modelado de bases de datos no relacionales. Por lo que la utilizaci�n de modelos escritos en UML es ventajosa ya que UML es un lenguaje de modelado bien conocido por toda la comunidad de dise�adores.

Como se comenta al inicio del documento, el objetivo de este proyecto de fin de carrera es la implementaci�n de un generador de c�digo que transforme un modelo UML en su correspondiente c�digo ejecutable en Cassandra. Previa implementaci�n del generador de c�digo habr� que transformar el modelo UML a un modelo escrito en Cassandra. Para ello necesitaremos contar con el metamodelo de Cassandra-CQL y el metamodelo de UML. Dicho generador y dicha transformaci�n se desarrollaran utilizando el enfoque y las t�cnicas que proporcionan el Desarrollo Dirigido por Modelos. Esta implementaci�n ser� realizada con el framework que proporciona eclipse de modelado llamado EMF.

Como resultado del proyecto se genera un c�digo escrito en el lenguaje de Cassandra (CQL-Cassandra Query Language) que puede ser ejecutado en cualquier herramienta que soporte dicho lenguaje.


