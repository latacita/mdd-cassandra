%%==================================================================%%
%% Author : Abascal Fern�ndez, Patricia                             %%
%%          S�nchez Barreiro, Pablo                                 %%
%% Version: 1.3, 18/06/2013                                         %%                                                                                    %%                                                                  %%
%% Memoria del Proyecto Fin de Carrera                              %%
%% Archivo ra�z                                                     %%
%%==================================================================%%

\cdpchapter{Resumen}

En la �ltima d�cada, ciertas aplicaciones surgidas a ra�z de la expansi�n de internet, tales como Twitter o Amazon, han impuesto una serie de nuevos requisitos, como altas disponibilidades o tiempo de respuestas concretos para situaciones muy particulares, que no pueden ser satisfechos mediante el uso de los gestores de bases de datos relacionales tradicionales. Con objeto de satisfacer estos nuevos requisitos han ido apareciendo una serie de nuevas tecnolog�as de almacenamiento, denominadas gen�ricamente NoSQL (\emph{Not Only SQL}). Dichas tecnolog�as sacrifican ciertos aspectos de los sistemas relaciones tradicionales, como la ausencia de redundancia, con el objetivo de mejorar otros aspectos, como el tiempo de consulta o de escritura de un datos.

Hasta ahora las tecnolog�as NoSQL han proporcionado sistemas gestores de almacenamiento, lo que permite trabajar exclusivamente con ellos a nivel de implementaci�n. Por tanto, no existen actualmente procesos de desarrollo bien definidos que permitan generar implementaciones de modelos de datos NoSQL a partir de modelos conceptuales de alto nivel. 

El objetivo de este proyecto es crear, haciendo uso de modernas t�cnicas de desarrollo software dirigido por modelos, una herramienta que permita transformar un modelo de datos conceptual de alto nivel expresado en UML 2.0 en una implementaci�n para Cassandra, un sistema de almacenamiento de datos NoSQL basado en columnas.

\paragraph{Palabras Clave} \ \\

Desarrollo Dirigido por Modelos, Ingenier�a Dirigida por Modelos, Generaci�n de C�digo, Cassandra, Epsilon, UML, CQL.



