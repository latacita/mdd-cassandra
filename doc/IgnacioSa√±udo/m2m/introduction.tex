%%==================================================================%%
%% Author : Sa�udo Olmedo, Ignacio                                  %%
%% Author : S�nchez Barreiro, Pablo                                 %%
%% Version: 1.4, 21/06/2014                                         %%
%%                                                                  %%
%% Memoria del Proyecto Fin de Carrera                              %%
%% M2M/Introduccion                                                 %%
%%==================================================================%%


A la hora de desarrollar el generador de c�digo necesitamos establecer en primer lugar cuales van a ser los modelos de origen y de salida, en este proyecto de fin de carrera y como se especifica en cap�tulos anteriores uno de los pasos previos a la construcci�n del generador de c�digo consiste en transformar un modelo UML en un modelo Cassandra para ello es necesario contar con un meta-modelo definido en Cassandra y un meta-modelo de UML en cuanto al meta-modelo origen basado en UML 2.0 utilizaremos el que nos proporciona la propia plataforma Epsilon. Para la construcci�n del meta-modelo de Cassandra utilizaremos EMF. Este tema es abordado en la siguiente secci�n de este cap�tulo.

El siguiente paso consiste en establecer una serie de reglas de correspondencia entre el modelo de entrada y el modelo de salida. Estas reglas han de contemplar los distintos tipos de elementos que pueden aparecer en ambos lenguajes, as� como elementos que no aparecen en uno de los dos lenguajes y tiene importancia en el otro. La Secci�n 3.3 describe dichas reglas y parte del c�digo desarrollado para llevarlas a cabo.
En este caso, consiste en establecer reglas de correspondencia entre elementos del modelado UML 2.0 y elementos del modelado Cassandra, para as� crear el generador de c�digo Cassandra Query Language (CQL).
Para la definici�n de estas reglas se ha utilizado el lenguaje Epsilon Transformation Language (ETL) y el lenguaje Epsilon Object Language (EOL) los cuales nos proporciona la plataforma Epsilon. 
Tras implementar estas reglas y comprobar que son funcionales se puede implementar el generador de c�digo pero este tema se trata en el siguiente capitulo.

El presente cap�tulo describe este proceso de desarrollo. En resumen, la Secci�n 3.2 describe como se ha realizado el lenguaje de modelado de Cassandra (definici�n del meta-modelo). La Secci�n 3.3 presenta las correspondencias definidas ente los elementos UML 2.0 y elementos de Cassandra as� como el c�digo necesario para llevar a cabo estas correspondencias, finalmente la secci�n 3.4 muestra el caso de estudio presentado en el cap�tulo anterior. 