%%==================================================================%%
%% Author : Sa�udo Olmedo, Ignacio                                  %%
%% Author : S�nchez Barreiro, Pablo                                 %%
%% Version: 1.5, 15/05/2014                                         %%
%%                                                                  %%
%% Memoria del Proyecto Fin de Carrera                              %%
%% m2t/Sumario                                                      %%
%%==================================================================%%

Durante este cap�tulo se ha descrito todo el proceso \imp{Model to Model} (M2M) realizado. En primer lugar se ha descrito la construcci�n del meta-modelo de Cassandra, proceso base para la transformaci�n entre modelos.
A continuaci�n se han presentado cuales son las reglas que se han utilizado a la hora de transformar cada uno de los elementos de un modelo UML a un modelo Cassandra. Para lograr este objetivo se han definido una serie de reglas escritas en el lenguaje Epsilon Transformation Language (ETL), de esta manera podemos transformar un modelo UML en un modelo Cassandra. La siguiente secci�n est� dedicada a continuar el caso de estudio desarrollado a lo largo del proyecto relacionado con Twissandra, en esta secci�n se explican como se transforman los elementos que aparecen en el modelo UML de Twissandra a el modelo de Cassandra. 