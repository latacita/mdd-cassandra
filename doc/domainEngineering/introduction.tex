%%==================================================================%%
%% Author : Abascal Fern�ndez, Patricia                             %%
%% Author : S�nchez Barreiro, Pablo                                 %%
%% Version: 1.4, 21/06/2013                                         %%
%%                                                                  %%
%% Memoria del Proyecto Fin de Carrera                              %%
%% Domain Engineering/Transformaci�n UML a C#                       %%
%%==================================================================%%

El primer paso a la hora de desarrollar un generador de c�digo es establecer una serie de correspondencias entre los distintos tipos de elementos que pueden aparecer en los modelos que sirven como entrada y los elementos del lenguaje de programaci�n destino. En nuestro caso, se trata de establecer una correspondencia entre elementos UML 2.0 y el lenguaje C\#, teniendo en cuenta que los elementos de entrada como los de salida deben seguir un enfoque orientado a caracter�sticas. Para el caso concreto de la implementaci�n, se debe hacer uso, de acuerdo a los objetivos iniciales del proyecto, del \emph{Slicer Pattern}. 

Una vez que las correspondencias estuviesen claramente definidas, el siguiente paso era implementar las plantillas de generaci�n de c�digo que deb�an establecer dichas correspondencias. El proceso de implementaci�n de estas plantillas no es trivial. Ello se debe b�sicamente a que los proceso de generaci�n de c�digo, al generar texto, son secuenciales. Es decir, una vezx generado un elemento, no podemos volver atr�s para modificarlo. Por ejemplo, si tras generar la cabecera de una clase descubri�semos que debemos a�adir una relaci�n de herencia a dicha clase, no podr�amos realizar tal modificaci�n. Por ello, cuando se implementan generadores de c�digo, hay que prestar especial atenci�n a su secuenciaci�n. 

Tras implementar los generadores de c�digo, el siguiente paso es realizar las pruebas que correspondan para poder comprobar el correcto funcionamiento de los generados de c�digo implementados. Para ello se deb�an dise�ar un conjunto de pruebas de forma sistem�tica, y ejecutar dicho conjunto de pruebas para verificar el correcto funcionamiento de los generadores de c�digo creados. La Secci�n~\ref{domain:sec:pruebas} describe dicho proceso de dise�o y ejecuci�n de las pruebas. 

Este cap�tulo describe este proceso de desarrollo. M�s concretamente, la Secci�n~\ref{domain:sec:transf} describe las correspondencias definidas ente elementos UML 2.0 y elementos de C\#. La Secci�n~\ref{domain:sec:gen} describe la secuenciaci�n de las plantillas de generaci�n de c�digo, mientras que la Secci�n~\ref{domain:sec:ejsencillo} muestra a modo de ejemplo, una sencilla plantilla de generaci�n de c�digo.



