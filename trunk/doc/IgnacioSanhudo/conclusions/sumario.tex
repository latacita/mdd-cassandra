%%==================================================================%%
%% Author : Abascal Fern�ndez, Patricia                             %%
%% Author : S�nchez Barreiro, Pablo                                 %%
%% Version: 1.5, 15/05/2013                                         %%
%%                                                                  %%
%% Memoria del Proyecto Fin de Carrera                              %%
%% Antecedentes, Sumario                                            %%
%%==================================================================%%

Durante este cap�tulo se han descrito los conceptos necesarios para comprender el �mbito y el alcance de este proyecto. Se ha descrito el caso de estudio que se utilizar� a lo largo del documento adem�s se ha especificado qu� es una l�nea de productos software y la tecnolog�a TENTE utilizada para su desarrollo, el dise�o orientado a caracter�sticas UML, la respuesta a las limitaciones de las clases parciales en C\# mediante el uso del patr�n slicer y la generaci�n de c�digo con Epsilon.

En el siguiente cap�tulo profundizaremos acerca de la fase dedicada a la implementaci�n de los generadores de c�digo que se corresponde con la fase de \emph{Ingenier�a del Dominio} del desarrollo de l�neas de producto software.
