%%==================================================================%%
%% Author : Abascal Fern�ndez, Patricia                             %%
%%          S�nchez Barreiro, Pablo                                 %%
%% Version: 1.3, 18/06/2013                                         %%                                                                                    %%                                                                  %%
%% Memoria del Proyecto Fin de Carrera                              %%
%% Archivo ra�z                                                     %%
%%==================================================================%%

\cdpchapter{Resumen}
Dentro del Departamento de Matem�ticas, Estad�stica y Computaci�n se han desarrollado con anterioridad una serie de t�cnicas para la implementaci�n y configuraci�n de l�neas de productos software para la plataforma .NET bas�ndose en las clases parciales de C\#. Dichas t�cnicas se condensan en el denominado \emph{Slicer Pattern}. No obstante, la aplicaci�n de dicho patr�n de forma manual implica una serie de tareas manuales y repetitivas.

El objetivo de presente proyecto fin de carrera es desarrollar una serie de generadores de c�digo que permitan automatizar la aplicaci�n del \emph{Slicer Pattern}. Ello reducir�a los tiempos de desarrollo; y , por tanto, el coste. Adem�s, al automatizarse el proceso se evita la introducci�n de errores debidos a la intervenci�n humana. Esto contribuye a aumentar la calidad del producto final y a reducir los tiempos y costes de desarrollo; ya que el tiempo necesario para detectar y corregir estos potenciales errores desaparece.

Para alcanzar dicho objetivo, este proyecto fin de carrera ha desarrollado una serie de generadores de c�digo que transforman modelos de dise�o, en UML 2.0, de una l�nea de productos software en una implementaci�n en C\# basada en el \emph{Slicer Pattern}. Dichos generadores de c�digo se han implementado usando EGL (\emph{Epsilon Generation Language}), el lenguaje de transformaci�n modelo a c�digo de la suite de herramientas para la manipulaci�n de modelos \emph{Epsilon}.

\paragraph{Palabras Clave} \ \\

L�nea de Productos Software, Generaci�n de C�digo, Desarrollo Software Orientado a Caracter�sticas, Clases Parciales C\#, Patr�n Slicer, .NET, Epsilon, Te.NET, TENTE.



