%%==================================================================%%
%% Author : Sa�udo Olmedo, Ignacio                                  %%
%%          S�nchez Barreiro, Pablo                                 %%
%% Version: 2.2, 18/06/2013                                         %%                                                                                    %%                                                                  %%
%% Memoria del Proyecto Fin de Carrera                              %%
%% Introducci�n                                                     %%
%%==================================================================%%

El principal objetivo de este proyecto es la implementaci�n de un generador de c�digo Cassandra-CQL (Cassandra Query Language) a partir de modelos UML, para ello se utilizaran una serie de t�cnicas orientadas al desarrollo software dirigido por modelos, estas reglas han sido definidas por los profesores Pablo S�nchez Barreiro y Carlos Blanco Bueno de la Universidad de Cantabria.

Uno de los aspectos  m�s valorados por las empresas dedicadas al desarrollo software hoy en d�a es la reducci�n de costes y tiempo en los proyectos de desarrollo software. Orientado a otro �mbito este problema tambi�n se planteo Henry Ford cuando propuso el paradigma basado en cadena de montaje a principios del siglo XX (1908). Henry Ford propuso una serie de t�cnicas y est�ndares para la automatizaci�n de fabricaci�n de sus veh�culos,  este m�todo lo puso en marcha con el conocido modelo Ford T. Con esto dio comienzo a la era de la producci�n industrial en serie.

A�os m�s tarde el t�rmino conocido como crisis del software fue concebido por Friedrich Ludwig Bauer [1] en la cumbre de la OTAN en el a�o 1968, en esta cumbre se debatieron cuales eran los problemas que se iban detectando en los proyectos de desarrollo software, entre ellos podemos encontrar los siguientes: Proyectos que no cumplen los presupuestos, ineficiencia del software, software que no cumple los requisitos, c�digo dif�cil de mantener y componentes no reutilizables.

El paradigma planteado por Henry Ford basado en la cadena de montaje puede ser aplicado al campo del desarrollo software por lo tanto el objetivo planteado para subsanar algunos de los problemas citados anteriormente en este �rea fue tratar de trasladar el concepto de la cadena de montaje al desarrollo software, esto es automatizar los procesos de producci�n para obtener l�neas de productos software y poder reutilizar componentes software que ya fueron desarrollados ya que el paradigma tradicional se basa en el redise�o de componentes usados con anterioridad.

A ra�z de este concepto surge el  denominado "Desarrollo Dirigido por Modelos" (MDD). MDD se puede definir como  un enfoque de la Ingenier�a del software y de la Ingenier�a dirigida por modelos (MDE) que utiliza el modelo para crear un producto. �Y que es un modelo?. Un modelo se puede entender como la descripci�n o representaci�n de un sistema en un lenguaje bien definido.
Para entender lo que representa un modelo dentro de la Ingenier�a dirigida por modelos (MDE) hay que saber previamente lo que es un metamodelo. Un metamodelo es un modelo usado para especificar un lenguaje, b�sicamente describe las caracter�sticas del lenguaje. Por lo tanto un modelo se puede entender como la instancia de un metamodelo. Estos conceptos son ampliados en siguientes secciones.

El resultado de la utilizaci�n de MDD es traducido en reducci�n de costes debido a que el recurso humano requerido es menor,  aumento de la productividad  y reutilizaci�n de componentes adem�s se puede aumentar el nivel de abstracci�n a la hora de realizar el dise�o del software.

En las siguientes secciones se desarrollan los siguientes aspectos: El apartado 1.2 expande informaci�n sobre la Ingenier�a dirigida por modelos y el Desarrollo Dirigido por Modelos, este apartado es vital para entender todo lo relacionado con la memoria presente. El apartado 1.3 presenta la motivaci�n y objetivos del proyecto. Finalmente el apartado 1.4 describe la estructura que tendr� el documento presente.








