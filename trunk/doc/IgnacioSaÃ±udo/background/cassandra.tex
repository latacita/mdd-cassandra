%%==================================================================%%
%% Author : Sa�udo Olmedo, Ignacio                                  %%
%%          S�nchez Barreiro, Pablo                                 %%
%% Version: 1.1, 18/06/2014                                         %%
%%                                                                  %%
%% Memoria del Proyecto Fin de Carrera                              %%
%% Background/Cassandra                                             %%
%===================================================================%%
http://www.nosql-database.org/
[chalmers]
%http://www.acens.com/wp-content/images/2014/02/bbdd-nosql-wp-acens.pdf
%http://www.strozzi.it/cgi-bin/CSA/tw7/I/en_US/NoSQL/Philosophy%20of%20NoSQL
Desde que naci� SQL en el a�o 1974, SQL ha sido el modelo de base de datos relacionales utilizado por excelencia. En los �ltimos a�os ha surgido otra vertiente denominada NoSQL la cual surge por la necesidad de manejo de grandes vol�menes de informaci�n no estructurada, distribuida con la mayor rapidez posible. NoSQL es usado en plataformas como Facebook o Twitter. Podemos encontrar varios tipos de bases de datos no relacionales; Bases de datos documentales, bases de datos clave-valor, orientadas a grafos y mas tipos.

Las principales caracter�sticas de NoSQL que difieren de SQL son:
\begin{enumerate}
\item NoSQL no garantiza las propiedades ACID (atomicidad, coherencia, aislamiento y durabilidad).
\item No utiliza SQL como lenguaje de consultas. Algunas bases de datos no relacionales utilizan SQL como lenguaje de apoyo sin embargo la mayor�a utilizan su propio lenguaje de consultas como por ejemplo Cassandra que utiliza CQL.
\item No est� permitido el uso de joins ya que al manejar grandes vol�menes de informaci�n una consulta con un join puede llegar a sobrecargar el sistema.
\item Escalan horizontalmente y trabajan de manera distribuida por lo que la informaci�n puede estar en distintas maquinas y el a�adir nodos mejora el rendimiento.
\item Resuelven problemas de altos vol�menes de informaci�n
\end{enumerate}

En este proyecto de fin de carrera se utiliza Cassandra, Cassandra es una distribuci�n de bases de datos no relaciones (NoSQL). Dentro del mundo de las bases de datos no relaciones Cassandra pertenece a la familia de bases de datos denominada "clave-valor". Las bases de datos clave-valor son aquellas que  asocian los valores a una determinada clave esto permite la recuperaci�n y escritura de informaci�n de forma muy r�pida y eficiente. Cassandra re�ne las  tecnolog�as de sistemas distribuidos de Amazon Dynamo y el modelo de datos BigTable de Google. Al igual que Dynamo, Cassandra es consistente. Como BigTable, Cassandra proporciona un modelo de datos basado en ColumnFamily siendo considerado el sistema basado en clave-valor m�s popular.

Las bases de datos basadas en Cassandra son soluciones utilizadas cuando es necesaria escalabilidad y alta disponibilidad sin comprometer el rendimiento. Escalabilidad lineal y tolerancia a fallos o infraestructura en la nube lo convierten en la plataforma perfecta para datos de misi�n critica. Cassandra proporciona gran estabilidad en cuanto a la replicaci�n de datos a trav�s de m�ltiples datacenters, consiguiendo una menor latencia para sus usuarios y la tranquilidad de que no existan perdidas de datos ante ca�das. Cassandra utiliza un lenguaje llamado CQL (Cassandra Query Language) con una sintaxis muy similar a SQL aunque con menos funcionalidades.

Fue dise�ado por Avinash Lakshman (uno de los creadores de Amazon's Dynamo) y Prashant Malik (Ingeniero de Facebook). Cassandra est� en producci�n para Facebook sin embargo aun est� en fase de desarrollo.

%https://cassandra.apache.org/ http://www.datastax.com/documentation/cql/3.0/cql/ddl/ddl_intro_c.html

A continuaci�n se explican algunos t�rminos que hay que tener en cuenta cuando se trabaja con Cassandra.
En Cassandra un KeySpace es el equivalente a una base de datos en los sistemas de bases de datos relacionales. El conocido t�rmino de tabla en las bases de datos relacionales tiene su equivalente en Cassandra llamado Column Family. Por lo tanto un KeySpace puede contener varias Column Families y una Column Family a su vez contiene varias columnas.
Una columna es la unidad de almacenamiento b�sica, est� formada de tres campos: Nombre, un valor y un timestamp. El nombre y el valor se almacena como una matriz de bytes sin procesar y pueden ser de cualquier tama�o. Un ejemplo:
Nombre de la columna 	\"Username\"
Nombre de usuario	\"Ignacio\"
Timestamp 		\"123456789\"
La sintaxis del lenguaje CQL es muy similar a SQL, CQL contiene sintaxis ya conocida de SQL como INSERT, DELETE, UPDATE, INSERT,  ...
Un ejemplo de c�digo CQL ejecutable es el siguiente (Figura~\ref{back:code:codigoCQL})


\begin{figure}[!tb]
\begin{center}
\begin{footnotesize}
\begin{verbatim}

DROP KEYSPACE twitter;

CREATE KEYSPACE twitter
WITH replication = {'class':'SimpleStrategy', 'replication_factor':2};

USE twitter;

CREATE TABLE Tweet(
       UUID text,
       usernameTw text,
       body text,
       PRIMARY KEY(UUID)
);

CREATE TABLE User(
       username text,
       password text,
       followers set<text>,
       followings set<text>,
       tweets_written list<text>,
       PRIMARY KEY(username)
);

\end{verbatim}
\end{footnotesize}
\end{center}
\caption{Ejemplo c�digo CQL}
\label{back:code:codigoCQL}
\end{figure}


