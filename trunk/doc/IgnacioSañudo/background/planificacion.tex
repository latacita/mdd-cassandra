%%==================================================================%%
%% Author : Sa�udo Olmedo, Ignacio                                  %%
%%          S�nchez Barreiro, Pablo                                 %%
%% Version: 1.2, 18/06/2013                                         %%
%%                                                                  %%
%% Memoria del Proyecto Fin de Carrera                              %%
%% Background/Planificacion                                         %%
%===================================================================%%

El objetivo de este proyecto de fin de carrera es la implementaci�n de un generador de c�digo Cassandra a partir de modelos UML. El proceso de desarrollo as� como el de aprendizaje que se ha seguido para la realizaci�n del proyecto es descrito a continuaci�n.

La primera tarea como es evidente consisti� en adquirir los conocimientos necesarios para el desarrollo del proyecto. En primer lugar todo lo relacionado con el proceso de modelado de un lenguaje y transformaci�n de lenguajes (\cite{kleppe:2008}). Tambi�n fueron necesarios conocimientos sobre la Ingenier�a y el Desarrollo Dirigido por Modelos, as� como de la sintaxis, arquitectura y funcionamiento de Cassandra.

A continuaci�n se comenz� a trabajar con la herramienta Epsilon, sus lenguajes EOL, EGL, ETL y finalmente EUnit como herramienta para las pruebas de los modelos y c�digo generados. Adem�s el lenguaje para la definici�n de lenguajes de modelado Eclipse Modeling Framework (EMF).
Para conocer c�mo funcionaban estos lenguajes se desarrollaron una serie de casos pr�cticos para familiarizarse con los m�todos de transformaci�n as� como con la herramienta, para ello se realizo el proceso completo para crear un generador de c�digo desde la transformaci�n entre modelos hasta la transformaci�n modelo-c�digo. Estos casos de prueba son los que se exponen en la secci�n de Epsilon (secci�n 2.3).

Una vez conocido estos conceptos se estudiaron las reglas de transformaci�n a aplicar para transformar un modelo UML a un modelo Cassandra, estas reglas fueron propuestas por \cite{pablo:2013}.

Tras estas tareas de adquisici�n de conocimientos se comenz� a trabajar en el generador de c�digo Cassandra empezando por la transformaci�n de modelos UML a modelos Cassandra. Estas transformaciones son expuestas en el siguiente cap�tulo. Una vez finalizada la transformaci�n se comenz� a trabajar en el generador de c�digo Cassandra, esta tarea es descrita en el cap�tulo 5. Tras realizar dicha tarea se realizaron una serie de casos de prueba para verificar si los resultados que otorgaba el generador de c�digo eran los esperados, para esta tarea utilizamos la herramienta EUnit.
Finalmente y tras generar varios casos de ejemplo se instalo DataStax OpsCenter y Apache Cassandra, se ejecutaron los resultados generados y se comprob� su correcto funcionamiento. 