%%==========================================================================%%
%% Author : Sa�udo Olmedo, Ignacio                                          %%
%% Author : S�nchez Barreiro, Pablo                                         %%
%% Version: 1.2, 21/04/2014                                                 %%
%%                                                                          %%
%% Memoria del Proyecto Fin de Carrera                                      %%
%% M2M/Caso de estudio                                                      %%
%%==========================================================================%%

Como se explicaba en el capitulo 2 en la secci�n "Caso de estudio" el objetivo de este caso de estudio es la creaci�n de un generador de c�digo de una versi�n simplificada de Twitter.
En esta secci�n se reproducir�n los procesos M2M y M2T en el siguiente capitulo, todo esto bajo el proceso de desarrollo dirigido por modelos. Esta secci�n esta dedicada a describir la transformaci�n del modelo UML de Twissandra a un modelo Cassandra. Partiendo del modelo UML de la figura~\ref{back:fig:twissandra} y una vez establecidas las reglas de transformaci�n entre modelos, esta secci�n explica el proceso de transformaci�n del modelo UML al modelo Cassandra.

En primer lugar, marcamos el atributo username de la clase User como clave. En el caso de la clase FollowingRelationship y la clase Tweet al no tener un atributo marcado como clave generamos una clave sustituta autom�ticamente para cada clase, llamadas FollowingRelationship\_id y tweet\_id respectivamente.

Por cada paquete estereotipado como <<dataModel>>, se crea un nuevo keyspace. El nombre del keyspace ser� el nombre utilizado por el data model. Los atributos restantes de las meta-clases del keyspace se establecen en sus valores definidos por defecto. A continuaci�n, todos los elementos correspondientes de ese paquete se procesan y se colocan dentro de su keyspace correspondiente.

Como se mencionaba en las reglas de transformaci�n, la clase User se transforma en una Column Family llamada User. A continuaci�n, los atributos y las asociaciones se procesan. Por �ltimo, el atributo Username, que se marc� como clave en el modelo de datos UML, es marcado como Primary Key. De manera similar para aquellos atributos del modelo UML cuya multiplicidad sea igual a uno se realiza una transformaci�n simple, por ejemplo el atributo username y password se transforman en dos columnas, ambos del tipo text. Estas columnas est�n contenidas en la column family User. De la misma manera se transforman los atributos body y time de la clase Tweet y el atributo timestamp de la clase FollowingRelationship.
En el caso de el atributo del modelo UML email cuya multiplicidad es mayor de uno y tiene las propiedades isUnique establecida en false y la propiedad isOrdered establecida en false (en el modelo no se puede apreciar pero esta configurado as�), se transforma este atributo en un set llamado email de tipo text dentro de la column family User.

En cuanto a las asociaciaciones de multiplicidad igual a uno la transformaci�n que se realiza por ejemplo, para la asociaci�n de la clase user una nueva columna llamada user\_username (recordemos que username es la clave de la column family user) es creada y a�adida a la column family Tweet. Para las asociaciones cuya multiplicidad es mayor de uno, por ejemplo la asociaci�n llamada userline se crea una dynamic column family llamada User\_userline. A continuaci�n una columna llamada user\_username de tipo text es a�adida a esta column family. Despu�s una columna llamada tweet\_id de tipo uuid es a�adida (el atributo tweet\_id fue creado en la column family tweet al no tener clave). Las columnas user\_username y tweet\_id son designadas como primary key, la columna user\_username ser� la partition key. 