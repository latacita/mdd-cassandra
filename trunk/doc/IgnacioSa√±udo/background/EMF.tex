%%==================================================================%%
%% Author : Sa�udo Olmedo, Ignacio                                  %%
%%          S�nchez Barreiro, Pablo                                 %%
%% Version: 1.3, 18/06/2014                                         %%
%%                                                                  %%
%% Memoria del Proyecto Fin de Carrera                              %%
%% Background/EMF                                                   %%
%===================================================================%%
Para comenzar el desarrollo del proyecto bajo el paradigma del desarrollo software dirigido por modelos necesitamos definir que lenguajes de modelado vamos a utilizar. En el cap�tulo anterior coment�bamos que necesitamos definir el lenguaje de modelado Cassandra para ello utilizaremos EMF. Eclipse Modeling Framework (EMF) [2.2] es un framework de modelado que nos proporciona la base para la elaboraci�n de lenguajes de modelado. Para la creaci�n de meta-modelos EMF [2.3] utiliza dos modelos de meta-datos: Ecore y Genmodel. Ecore contiene la informaci�n sobre las clases que se han definido. Genmodel contiene informaci�n adicional para la generaci�n del c�digo, por ejemplo la ruta y la informaci�n del archivo. Genmodel tambi�n contiene atributos que sirven de control a la hora de generar el c�digo, encontramos los siguientes par�metros de control:
\begin{enumerate}
\item EClass: representa una clase, con cero o m�s atributos y cero o m�s referencias.
\item EAttribute: representa un atributo que tiene un nombre y un tipo.
\item EReference: representa un extremo de una asociaci�n entre dos clases.
\item EDataType: representa el tipo de un atributo, por ejemplo, int, float.
\end{enumerate}
Estos atributos nos servir�n a la hora de crear las reglas de transformaci�n entre modelos UML y modelos Cassandra.

Utilizando EMF se pueden crear meta-modelos de forma gr�fica muy similar a los diagramas de clases en UML.
EMF permite crear un meta-modelo a trav�s de diferentes medios, por ejemplo, XMI, anotaciones Java, XML o UML. Adem�s EMF proporciona un framework para almacenar la informaci�n del modelo.
Un ejemplo sencillo de la utilidad de EMF es el siguiente: Imaginemos que deseamos construir una aplicaci�n para manipular mensajes escritos en XML. El primer paso que dar�amos ser�a empezar definiendo el schema del mensaje sin embargo con EMF se puede trabajar ignorando este nivel. Con EMF podemos crear plugins que generen por ejemplo un diagrama de clases UML a partir de este mensaje XML o directamente generar el c�digo Java que implemente las clases del mensaje XML.


