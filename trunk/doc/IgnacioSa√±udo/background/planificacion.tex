%%==================================================================%%
%% Author : Sa�udo Olmedo, Ignacio                                  %%
%%          S�nchez Barreiro, Pablo                                 %%
%% Version: 1.2, 18/06/2013                                         %%
%%                                                                  %%
%% Memoria del Proyecto Fin de Carrera                              %%
%% Background/Planificacion                                         %%
%===================================================================%%

Como se ha comentado en el presente documento, el objetivo de este proyecto de fin de carrera es la implementaci�n de un generador de c�digo Cassandra a partir de modelos UML. El proceso de desarrollo as� como el de aprendizaje que se ha seguido para la realizaci�n del proyecto queda reflejado en la figura [figuraProceso].

La primera tarea como es evidente consisti� en adquirir los conocimientos necesarios para el desarrollo del proyecto, en primer lugar todo lo relacionado con el proceso de modelado de un lenguaje y transformaci�n de lenguajes [kleppe]. Tambi�n fueron necesarios conocimientos sobre la Ingenier�a y el Desarrollo Dirigido por Modelos, tambi�n sintaxis y arquitectura
de Cassandra.

A continuaci�n se comenz� a trabajar con la herramienta Epsilon, sus lenguajes EOL, EGL, ETL y EUnit como herramienta para las pruebas de los modelos y c�digo generados.  As� como con lenguaje para la definici�n de lenguajes de modelado EMF.

Para conocer c�mo funcionaban estos lenguajes se desarrollaron una serie de casos pr�cticos para familiarizarse con los m�todos de transformaci�n as� como con la herramienta y creaci�n de casos de prueba.

Una vez conocido estos conceptos se estudiaron las reglas de transformaci�n de un modelo UML a un modelo Cassandra, estas reglas fueron propuestas por [pabloCassandra].

Tras estas tareas de adquisici�n de conocimientos se empez� a trabajar en el generador de c�digo Cassandra empezando por la transformaci�n de modelos UML a modelos Cassandra. Una vez finalizada la transformaci�n se comenz� a trabajar en el generador de c�digo Cassandra. Tras realizar dicha tarea se realizaron una serie de casos de prueba para verificar si el resultado del generador de c�digo era el esperado utilizando la herramienta EUnit.

Una vez desarrollado el generador de c�digo se realizaron distintos casos de prueba sobre Cassandra para probar el correcto funcionamiento del generador de c�digo.

