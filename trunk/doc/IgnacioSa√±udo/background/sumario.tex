%%==================================================================%%
%% Author : Sa�udo Olmedo, Ignacio                                  %%
%%          S�nchez Barreiro, Pablo                                 %%
%% Version: 1.1, 21/06/2014                                         %%
%%                                                                  %%
%% Memoria del Proyecto Fin de Carrera                              %%
%% Antecedentes, Sumario                                            %%
%%==================================================================%%

%%==================================================================%%
%% NOTA(Pablo): Esto habr�a que casarlo para que cuadre con el      %%
%%              cap�tulo actualizado                                %%
%%              La estructura ha cambiado                           %%
%%==================================================================%%

Durante este cap�tulo se han descrito los conceptos necesarios para lograr comprender el �mbito y el alcance de este proyecto, se ha descrito el caso de estudio planteado en el proyecto. Tambi�n se ha hablado sobre tecnolog�as implicadas en el desarrollo del generador de c�digo, as� como de Cassandra y su arquitectura, Epsilon los lenguajes utilizados y herramientas utilizadas. El siguiente cap�tulo describe los primeros pasos para la realizaci�n del generador de c�digo, el primer paso consiste en la definici�n de reglas de transformaci�n entre modelos llamadas \imp{Model to Model} (M2M). Adem�s se continua el caso de estudio planteado en este cap�tulo, se define el meta-modelo de Cassandra y se definen las reglas de transformaci�n entre modelos UML y Cassandra.
