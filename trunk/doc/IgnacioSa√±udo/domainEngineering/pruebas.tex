%%=======================================================================%%
%% Author : Abascal Fern�ndez, Patricia                                  %%
%% Author : S�nchez Barreiro, Pablo                                      %%   %%                                                                       %%
%% Version: 2.0, 25/06/2013                                              %%   %%                                                                       %%
%% Memoria del Proyecto Fin de Carrera                                   %%
%% Domain Engineering/Pruebas con EUnit                                   %%   %%=======================================================================%%

Una vez implementados los generadores de c�digo, la siguiente tarea era comprobar su correcto funcionamiento. Para ello creamos, de forma sistem�tica, una serie de pruebas unitarias que permitiesen comprobar el correcto funcionamiento de los generadores de c�digo para un exhaustivo conjunto de diferentes tipos de entrada. Estas pruebas unitarias se implementaron en \emph{EUnit}~\cite{kolovos:2008}, el lenguaje de definici�n de pruebas de la suite Epsilon. El funcionamiento de \emph{EUnit} es similar al de \emph{JUnit}, pero aplicado a los lenguajes de la suite Epsilon, como EGL.

Para comprobar que el funcionamiento del generador de c�digo es correcto, se dise�a el caso de prueba y se crea la salida de ese caso de prueba de forma manual. A continuaci�n, se ejecuta el caso de prueba y se comprueba que la salida generada coincide con la esperada, que es la creada manualmente. Al intentar implementar estas pruebas en \emph{EUnit}, nos encontramos con el problema inicial de que este lenguaje no ten�a implementada la comparaci�n de fragmentos de texto en ficheros, que era precisamente la funcionalidad que nos hac�a falta. No obstante, se curs� una petici�n a los creadores de Epsilon, los cuales, muy amablamente, incorporaron dicha funcionalidad a \emph{EUnit}. De igual forma, a�adieron otra funcionalidad para comprobar que dos directorios conten�an los mismos archivos (\imp{assertEqualDirectories}).

Para el dise�o de los casos de prueba se sigui� inicialmente un enfoque de caja negra, basado en una adaptaci�n de la t�cnica de clases de equivalencia y an�lisis de valores l�mite al entorno de los modelos software. Una vez ejecutadas estas pruebas, se analiz� la cobertura alcanzada, definiendo pruebas adicionales, ya de caja blanca, de forma que la cobertura alcanzada fuese del 100\%.

La Tabla~\ref{dom:table:prueba} resume algunos de los casos de prueba
ejecutados. Concretamente, se muestran los casos de prueba para analizar el comportamiento de las plantillas de generaci�n de c�digo para paquetes y clases.

\begin{table}%
\begin{small}
\begin{tabularx}{\linewidth}{|l|p{6cm}|l|}
 \hline
{}&{Casos v�lidos}&{Casos no v�lidos} \\ \hline
\multirow{4}{*}{Paquete} & Paquete con nombre. & Paquete sin nombre. \\
& Paquete con clases e interfaces en su interior. & \\
& Paquete vac�o. & \\
& Paquete dentro de otro paquete (recursividad). & \\
\hline
\multirow{12}{*}{Clase} & Clase con nombre. & Clase fuera de un paquete. \\
& Clase tipo abstract. & Clase sin nombre.\\
& Clase sin tipo. & \\
& Clase que hereda de una o varias clases. &\\
& Clase que hereda de una o varias interfaces. &\\
& Clase que hereda de clases e interfaces. &\\
& Clase sin propiedades. &\\
& Clase sin m�todos. &\\
& Clase sin propiedades ni m�todos. &\\
& Clase con propiedades. &\\
& Clase con m�todos. &\\
& Clase con propiedades y m�todos. &\\
\hline
\end{tabularx}
\end{small}
\caption{Casos de prueba para lo generadores de Ingenier�a del Dominio}
\label{dom:table:prueba}
\end{table}