%%==================================================================%%
%% Author : Sa�udo Olmedo, Ignacio                                  %%
%% Author : S�nchez Barreiro, Pablo                                 %%
%% Version: 1.5, 15/05/2014                                         %%
%%                                                                  %%
%% Memoria del Proyecto Fin de Carrera                              %%
%% m2t/Sumario                                                      %%
%%==================================================================%%

Esta memoria de Proyecto Fin de Carrera ha descrito el proceso de desarrollo de un generador de c�digo Cassandra bajo el desarrollo dirigido por modelos.

Como hemos visto el Desarrollo Dirigido por Modelos aporta multiples ventajas a los desarrolladores software, la automatizaci�n de los procesos de producci�n nos permite crear software m�s r�pido llegando a considerar a este paradigma como la verdadera industrializaci�n de la producci�n de software esto permite a las empresas ahorro en tiempo y costes.
La base de la utilizaci�n de Cassandra se sostiene en aspectos como la disponibilidad o el manejo de gran cantidad de datos, aspectos que no son facilitados mediante la utilizaci�n de bases de datos relacionales tradicionales. Adem�s dentro de las bases de datos no relacionales orientadas a columnas Cassandra es el sistema utilizado por excelencia.
En cuanto a la utilizaci�n de UML como lenguaje de modelado 'origen', consideramos que la utilizaci�n de modelos UML respecto a modelos dise�ados en Cassandra a la hora de crear una base de datos no relacional proporciona una abstracci�n para aquellos desarrolladores que no est�n muy familiarizados con el modelado de bases de datos no relacionales.

A nivel de implementaci�n, el desarrollo del generador de c�digo fue laborioso en t�rminos de adquisici�n de conceptos ya que el plan de estudios de la carrera de Ingenier�a Inform�tica aporta pocos conocimientos en este area. Para ello en primer lugar se estudiaron conceptos relacionados con el modelado de lenguajes [kleppe] para entender y poder trabajar con el meta-modelo de Cassandra, a continuaci�n todo lo vinculado con el Desarrollo y la Ingenier�a Dirigida por Modelos, t�cnicas de transformaci�n (M2M,M2T), ventajas etc.. Finalmente el estudio de todos los lenguajes y herramientas que ofrece la plataforma de Epsilon (ETL, EGL, EOL, EUnit..).
Una vez dominados estos conceptos se pusieron en practica todas las nociones aprendidas con una serie de ejemplos sencillos propuestos por el director del proyecto que han sido descritos a lo largo de esta memoria.

Tras estas tareas de estudio, se procedi� a la realizaci�n del principal objetivo del proyecto: la creaci�n de un generador de c�digo Cassandra que transforma modelos UML a modelos Cassandra. Para ello en primer lugar se hicieron una serie de cambios en el meta-modelo que nos proporcionaron de Cassandra mediante EMF. A continuaci�n se definieron las reglas de correspondencia entre modelos UML y Cassandra y se desarrollaron mediante el lenguaje ETL. Seguidamente y tras realizar el proceso de transformaci�n entre modelos (M2M), se desarrollaron las plantillas para la transformaci�n modelo-c�digo (M2T), para lograr esto se crearon plantillas para generar c�digo CQL, de esta manera obtuvimos el c�digo necesario para crear repositorios de datos en Cassandra.
Por �ltimo se realizaron una serie de casos de prueba con la herramienta EUnit para comprobar el correcto funcionamiento del generador de c�digo.

La siguiente secci�n describe las experiencias personales vividas a lo largo del proyecto.




