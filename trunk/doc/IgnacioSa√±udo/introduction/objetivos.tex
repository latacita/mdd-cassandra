%%==================================================================%%
%% Author : Sa�udo Olmedo, Ignacio                                  %%
%%          S�nchez Barreiro, Pablo                                 %%
%% Version: 1.4, 18/06/2014                                         %%                                                                                    %%                                                                  %%
%% Memoria del Proyecto Fin de Carrera                              %%
%% Introduccion/Motivaci�n y objetivos                              %%
%%==================================================================%%


Como hemos visto a lo largo de la introducci�n el enfoque que proporciona el Desarrollo Dirigido por Modelos as� como la Ingenier�a Dirigida por Modelos nos proporciona m�ltiples ventajas respecto al desarrollo tradicional por lo que la utilizaci�n de este enfoque es apropiada para la realizaci�n del generador de c�digo as� como para la transformaci�n entre modelos, adem�s la utilizaci�n de tecnolog�as NoSQL frente a tecnolog�as de gesti�n de bases de datos tradicionales en terrenos donde es de extrema importancia aspectos como la disponibilidad o el manejo de cantidades de datos descomunales garantizan que NoSQL es una apuesta de futuro segura.

El objetivo de este proyecto de fin de carrera es la implementaci�n de un generador de c�digo que transforme modelos especificados en UML 2.0 (~\cite{uml:2010}) en su correspondiente c�digo de creaci�n de bases de datos en Cassandra. Para ello, previa implementaci�n del generador de c�digo habr� que transformar el modelo UML a un modelo escrito en Cassandra, para lograr este objetivo ser� necesario definir una serie de reglas de equivalencias entre modelos utilizando para ello el lenguaje Epsilon Transformation Language (ETL). Por lo tanto necesitaremos desarrollar un meta-modelo que describa el modelado de Cassandra y un meta-modelo que describa el modelado de UML (este �ltimo meta-modelo lo proporciona la herramienta con la que vamos a trabajar). La construcci�n del meta-modelo se realizara utilizando el lenguaje para la definici�n de lenguajes de modelado Eclipse Modelling Framework (EMF).
Dicho generador y dicha transformaci�n se desarrollaran utilizando el enfoque y las t�cnicas que proporcionan el Desarrollo Dirigido por Modelos, para la creaci�n de este generador de c�digo utilizaremos el lenguaje Epsilon Generation Language (EGL). Por lo tanto utilizaremos la suite de lenguajes que proporciona Epsilon para el desarrollo de software dirigido por modelos.

Como resultado del proyecto se genera un c�digo para la creaci�n de una base de datos en Cassandra escrito en el lenguaje de consultas de Cassandra, Cassandra Query Language (CQL) que puede ser ejecutado en cualquier herramienta que soporte dicho lenguaje.



