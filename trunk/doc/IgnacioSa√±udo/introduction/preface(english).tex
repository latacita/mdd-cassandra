%%==================================================================%%
%% Author : Abascal Fern�ndez, Patricia                             %%
%%          S�nchez Barreiro, Pablo                                 %%
%% Version: 1.3, 18/06/2013                                         %%                                                                                    %%                                                                  %%
%% Memoria del Proyecto Fin de Carrera                              %%
%% Archivo ra�z                                                     %%
%%==================================================================%%

\cdpchapter{Preface}

Several applications, such as Twitter or Amazon, which have emerged as a consequence of the massive internet adoption, have starting to demand certain requirements, such as very high availability or reduced response times for particular situations. These new requirements can be hardly satisfied using traditional relational data management systems. Thus, new data storage systems, named NoSQL (\emph{Not Only SQL}) systems. These systems resign to certain properties of relational systems, such as being redundancy-free, in order to fulfill these new challenging requirements. 

Until now, several NoSQL data management systems have been released. This allows developers can work with NoSQL technologies at the implementation level, but they lack of development processes that assist them on the mapping of a data conceptual model into a NoSQL implementation. 

Thus, the goal of this project is to develop a tool that, using model-driven technologies, accepts as input a conceptual data model, described in UML 2.0, and automatically generates an implementation of this data model for Cassandra, a NoSQL column-oriented data store system.

\paragraph{Keywords} \ \\

Model-Driven Development,Model-Driven Engineering, Code Generation, Cassandra, Epsilon, UML, CQL.
