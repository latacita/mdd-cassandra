%%==========================================================================%%
%% Author : Sa�udo Olmedo, Ignacio                                          %%
%% Author : S�nchez Barreiro, Pablo                                         %%
%% Version: 1.2, 21/04/2014                                                 %%
%%                                                                          %%
%% Memoria del Proyecto Fin de Carrera                                      %%
%% M2M/Caso de estudio                                                      %%
%%==========================================================================%%

Como se explicaba en el cap�tulo 2 (secci�n "Caso de estudio"), el objetivo consiste en la creaci�n de un generador de c�digo de una versi�n simplificada de Twitter llamada Twissandra.
En esta secci�n se reproducir�n los procesos M2M y M2T en el siguiente cap�tulo, todo esto bajo el proceso de desarrollo dirigido por modelos. Esta secci�n est� dedicada a describir la transformaci�n del modelo UML de Twissandra a un modelo Cassandra, para ello partimos del modelo UML de la figura~\ref{back:fig:twissandra}. Una vez establecidas las reglas de transformaci�n entre modelos podemos realizar la generaci�n del c�digo aunque esto se analizar� en el siguiente cap�tulo.

En primer lugar, por cada paquete estereotipado como <<dataModel>>, se crea un nuevo keyspace. El nombre del keyspace ser� el nombre que se ha definido el modelo de datos UML. Los atributos restantes de las meta-clases del keyspace se establecen en sus valores definidos por defecto en el modelo UML. A continuaci�n, todos los elementos correspondientes de ese paquete se procesan.

A continuaci�n se realiza un marcado del atributo username de la clase User como clave, ya que el par�metro isID del modelo UML est� marcado como true. En el caso de las clases FollowingRelationship y la clase Tweet al no tener un atributo marcado como clave generamos dos columnas clave autom�ticamente para cada clase, llamadas FollowingRelationship\_id y tweet\_id respectivamente.
La clase User del modelo UML es transformada en una column family llamada User. Una vez procesadas las clases UML y transformarlas a su correspondiente column family se proceden a transformar los atributos y asociaciones. De manera similar para aquellos atributos del modelo UML cuya multiplicidad sea igual a uno se realiza una transformaci�n simple, por ejemplo el atributo username y password se transforman en dos columnas Cassandra, ambas del tipo text. Estas columnas est�n contenidas en la column family User. De la misma forma se transforman los atributos body y time de la clase Tweet y el atributo timestamp de la clase FollowingRelationship.
En el caso del atributo del modelo UML email cuya multiplicidad es mayor de uno y tiene las propiedades isUnique establecida en false y la propiedad isOrdered establecida en false (en el modelo no se puede apreciar pero est� configurado as� en el modelo UML), se transforma este atributo en una colecci�n de tipo set llamada email cuyo tipo primitivo ser� text, esta colecci�n estar� dentro de la column family User.

En cuanto a las asociaciaciones recordamos que tenemos dos tipos, las de multiplicidad igual a uno y las de multiplicidad mayor de uno. Para la asociaci�n de la clase User cuya multiplicidad es igual uno, se crea una nueva columna llamada user\_username (recordemos que username es la clave de la column family user) y esta columna es a�adida a la column family Tweet. Para las asociaciones de multiplicidad mayor de uno, por ejemplo la asociaci�n llamada userline se crea una dynamic column family llamada User\_userline. A continuaci�n una columna llamada user\_username de tipo text es a�adida a esta column family. Despu�s una columna llamada tweet\_id de tipo uuid es a�adida (el atributo tweet\_id fue creado en la column family tweet al no tener clave). Las columnas user\_username y tweet\_id son designadas como primary key, la columna user\_username ser� la partition key y la columna tweet\_id ser� la cluster key.


