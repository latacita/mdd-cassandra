%%==================================================================%%
%% Author : Sa�udo Olmedo, Ignacio                                  %%
%% Author : S�nchez Barreiro, Pablo                                 %%
%% Version: 1.2, 24/06/2014                                         %%
%%                                                                  %%
%% Memoria del Proyecto Fin de Carrera                              %%
%% m2t/Introduccion                                                 %%
%%==================================================================%%


Una vez establecidas las reglas de correspondencia entre ambos modelos empezamos a desarrollar el generador de c�digo. Este cap�tulo describe el proceso de desarrollo de los generadores de c�digo. El objetivo de esta
fase, tal como comentamos, es obtener un repositorio de datos NoSQL funcional a partir de un modelo de datos UML 2.0. El c�digo generado es c�digo Cassandra Query Language (CQL) que puede ser ejecutado en herramientas que soporten dicho lenguaje. Este proceso es descrito en la Secci�n 3.2 de este cap�tulo.
De manera an�loga al anterior cap�tulo la Secci�n 3.3 esta dedicada a aplicar esta transformaci�n modelo-c�digo al ejemplo de Twissandra.
A continuaci�n tras implementar el generador de c�digo el siguiente paso consiste en la implementaci�n de una serie de casos de prueba para comprobar el correcto funcionamiento del generador de c�digo. Esto es explicado en la Secci�n 3.4. Para la implementaci�n de este generador de c�digo se ha utilizado el lenguaje Epsilon Generation Language (EGL).

En resumen, la Secci�n 3.2 describe como se ha realizado el generador de c�digo. La Secci�n 3.3 continua el caso de estudio presentado en el cap�tulo 2, mientras que la Secci�n 3.4 describe las pruebas que se han realizado para comprobar el correcto funcionamiento del generador de c�digo. 