%%==================================================================%%
%% Author : Sa�udo Olmedo, Ignacio                                  %%
%% Author : S�nchez Barreiro, Pablo                                 %%
%% Version: 1.5, 15/05/2014                                         %%
%%                                                                  %%
%% Memoria del Proyecto Fin de Carrera                              %%
%% m2t/Sumario                                                      %%
%%==================================================================%%
Esta secci�n analiza el proceso de creaci�n del generador de c�digo.
En la figura~\ref{back:code:m2t} se muestra parte del c�digo del generador de c�digo. Este fragmento de c�digo muestra el lenguaje Epsilon Generation Language (EGL). El fragmento contiene en primer lugar la definici�n del Key Space. A continuaci�n se muestra parte del c�digo de la creaci�n de una Column Family en CQL. Por cada Column de la Column Family se va definiendo el tipo de dato (primitivo,map,set/list) junto con su nombre. El resto de c�digo se ha omitido por razones de espacio sin embargo se detalla a continuaci�n. Tras la definici�n de la columna se define la Primary Key. Para la definici�n de la Primary Key hay que diferenciar entre si la Column Family es din�mica o est�tica. En caso de ser est�tica la definici�n ser�a la cl�sica: PRIMARY KEY(ColumnaClave).
En caso de ser una Column Family din�mica la construcci�n modelo-c�digo es distinta, la cl�sica ser�a la siguiente: PRIMARY KEY(partitioning key, clustering key\_1 ... clustering key\_n)
Sin embargo hay que tener en cuenta que en la construcci�n tambi�n es posible tener una Partition Key compuesta, es decir, una Partition Key formada por varias columnas, para ello se utilizan par�ntesis y as� delimitar el conjunto de partici�n. Quedando la definici�n de la PRIMARY KEY de la siguiente manera: PRIMARY KEY((partitioning key\_1, ... partitioning key\_n), clustering key\_1 ... clustering key\_n)

\begin{figure}[!tb]
\begin{center}
\begin{footnotesize}
\begin{verbatim}
--------------------------------------------------------
DROP KEYSPACE [%=keyspace.name%];
CREATE KEYSPACE [%=keyspace.name%]
WITH replication = {'class':'[%=keyspace.replicaPlacementStrategy%]', 'replication_factor':[%=keyspace.replicationFactor%]};

USE [%=keyspace.name%];

[% for (cf in keyspace.columnFamilies){ tableKey="";%]
CREATE TABLE [%=cf.name%](
	[% for (cols in cf.columns){ //este for define cada grupo de columnas
		s=cols.name.toString();
		for (c in cols.type){
		
			if (c.isTypeOf(PrimitiveType))  //definicion del tipo primitivo
				s=s+" "+c.kind.toString();
				
			if (c.isTypeOf(MapType)) //definicion del tipo map
				s=s+" Map"+"<"+c.keyType.toString()+","+c.baseType.toString()+">";
				
 			if(c.isTypeOf(CollectionType)) //definicion del tipo set o list
				s=s+" "+c.kind.toString()+"<"+c.keyType.toString()+">";
			
		}
);
[%}%]
--------------------------------------------------------
\end{verbatim}
\end{footnotesize}
\end{center}
\caption{Regla de transformaci�n XX}
\label{back:code:m2t}
\end{figure} 