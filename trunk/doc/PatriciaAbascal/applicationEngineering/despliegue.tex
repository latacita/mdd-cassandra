%%==================================================================%%
%% Author : Abascal Fern�ndez, Patricia                             %%
%% Author : S�nchez Barreiro, Pablo                                 %%
%% Version: 1.5, 15/05/2013                                         %%
%%                                                                  %%
%% Memoria del Proyecto Fin de Carrera                              %%
%% Application Engineering/Despliegue                               %%
%%==================================================================%%

Una vez creados los generadores de c�digo, el siguiente paso es empaquetarlos y distribuirlos de forma que puedan ser usados de la forma m�s c�moda posible por diferentes desarrolladores para la creaci�n de productos concretos pertenecientes a nuestra familia de productos.

La forma m�s f�cil de distribuir nuestra infraestructura es crear un plugin,
que se integre con la plataforma Eclipse, que permita la generaci�n de un proyecto \emph{Visual Studio 2010} que contenga dos tipos de proyectos, en funci�n del generador de c�digo ejecutado. Se podr�a haber pensado en integrar los generadores de c�digo en \emph{Visual Studio 2010} o un editor comercial de UML, pero ninguna de dichas herramientas ofrec�a tales facilidades. Por ejemplo, desde \emph{Visual Studio 2010} resulta imposible invocar las plantillas de generaci�n de c�digo creadas. \emph{Visual Studio 2010} ofrece un lenguaje de generaci�n de c�digo, pero sus funcionalidades est�n a a�os luz de las ofrecidas por otros lenguajes de generaci�n de c�digo, como EGL.

En el caso de ejecutar el generador de c�digo correspondiente a la fase de \emph{Ingenier�a del Dominio}, se crear�a un proyecto que contendr�a los esqueletos de las clases parciales que conformar�an la \emph{implementaci�n de referencia}. Dichos esqueletos, como ya hemos comentado, han de completarse a mano. 

En el caso de ejecutar el generador de c�digo correspondiente a la fase de \emph{Ingenier�a de Aplicaciones}, se crear�a un proyecto que contendr�a las las clases parciales necesarias para componer las clases parciales adecuadas de la implementaci�n de referencia, de acuerdo a las caracter�sticas seleccionadas. 

Para integrar los generadores de c�digo en Eclipse, se ha hecho uso de su entorno \emph{Plug-in Development Environment} (PDE)~\citep{}. Utilizando este entorno, a�adimos una serie de elementos gr�ficos a Eclipse para permitir invocar las plantillas de generaci�n de c�digo desarrolladas. 

El c�digo asociado a dichos elementos gr�ficos realizaba varias tareas: (1) en primer lugar, se preprocesaba el modelo UML 2.0 para eliminar elementos innecesarios, como perfiles no utilizados, com�nmente introducidos por los editores comerciales. A continuaci�n, se invocan las plantillas EGL desde c�digo Java. 

Una vez desarrollado el plug-in, se procedi� a su empaquetado (siguiendo las directrices impuestas por Eclipse), y a la creaci�n del correspondiente instalador para Eclipse, conocido como \emph{update site}. Por �ltimo, se cre� una p�gina web para darle visibilidad al proyecto.