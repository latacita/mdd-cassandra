%%==================================================================%%
%% Author : Tejedo Gonz�lez, Daniel                                 %%
%%          S�nchez Barreiro, Pablo                                 %%
%% Version: 1.0, 27/11/2012                                         %%                   
%%                                                                  %%
%% Memoria del Proyecto Fin de Carrera                              %%
%% Gram�tica,  requisitos                                           %%
%%==================================================================%%

La captura de requisitos de la gram�tica pasa por informarse sobre qu� tipo de sintaxis textual queremos que tengan nuestras operaciones, lo cual en este caso ya estaba especificado previamente por documentos creados en iteraciones previas del proyecto Hydra. Lo m�s l�gico e inmediato era mantenerse fiel a esa sintaxis, seg�n la cual las operaciones deber�an ser expresadas textualmente tal como sigue: \\

Suma: operando1 + operando2

Resta: operando1 - operando2

Multiplicaci�n: operando1 * operando2

Divisi�n: operando1 / operando2

And: operando1 and operando2

Or: operando1 or operando2

Xor: operando1 xor operando2

Implica: operando1 implies operando2

Mayor que: operando1 > operando2

Menor que: operando1 < operando2

Igual que: operando1 == operando2

Mayor o igual que: operando1 >= operando2

Menor o igual que: operando1 <= operando2

Distinto que: operando1 != operando2

Contexto: operando1 [ operando2 ]

Para todo: all operando1 [ operando2 ]

Existe: any operando1 [ operando2 ] \\

Otros aspectos concernientes a la gram�tica que ha habido que tener en cuenta dentro de la fase de captura de requisitos son los siguientes:\\

- Ha de permitirse la posibilidad de especificar prioridad en las operaciones, es decir, de poder delimitar las operaciones con par�ntesis que denoten el orden de realizaci�n de las mismas.

- Todas las representaciones textuales de nuestro lenguaje han de empezar con una l�nea de carga del modelo de caracter�sticas al que han de aplicarse las restricciones. La sintaxis de este aspecto ser� "import operando", donde operando es la direcci�n del fichero hydra del modelo dentro del disco duro del sistema.

- Todas las restricciones definidas han de separarse entre ellas mediante el car�cter '' ; ''.\\

Por supuesto, adem�s de los requisitos aqu� expuestos tambi�n habr� que tener en cuenta todo lo comentado en el apartado de requisitos del cap�tulo anterior, ya que tambi�n influir�n a la hora de tomar decisiones de dise�o en la gram�tica.
